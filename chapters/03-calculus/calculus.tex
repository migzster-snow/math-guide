\chapter{Calculus}

This chapter covers the fundamental concepts of differential and integral calculus, building upon the foundations established in previous chapters.

\section{Limits and Continuity}

\begin{definition}[Limit]
Let $f$ be a function defined on some open interval containing $a$, except possibly at $a$ itself. We say that
\[\lim_{x \to a} f(x) = L\]
if for every $\varepsilon > 0$, there exists a $\delta > 0$ such that whenever $0 < |x - a| < \delta$, we have $|f(x) - L| < \varepsilon$.
\end{definition}

\begin{example}
To show that $\lim_{x \to 2} (3x - 1) = 5$:

Given $\varepsilon > 0$, we need $|(3x - 1) - 5| < \varepsilon$, which simplifies to $|3x - 6| = 3|x - 2| < \varepsilon$.

Taking $\delta = \frac{\varepsilon}{3}$, whenever $0 < |x - 2| < \delta$, we have:
\[|(3x - 1) - 5| = 3|x - 2| < 3 \cdot \frac{\varepsilon}{3} = \varepsilon\]
\end{example}

\begin{definition}[Continuity]
A function $f$ is \textbf{continuous} at $a$ if:
\[\lim_{x \to a} f(x) = f(a)\]
\end{definition}

\begin{theorem}[Intermediate Value Theorem]
If $f$ is continuous on $[a, b]$ and $k$ is any number between $f(a)$ and $f(b)$, then there exists a number $c \in (a, b)$ such that $f(c) = k$.
\end{theorem}

\section{Differentiation}

\begin{definition}[Derivative]
The \textbf{derivative} of a function $f$ at $a$ is:
\[f'(a) = \lim_{h \to 0} \frac{f(a + h) - f(a)}{h}\]
provided this limit exists.
\end{definition}

\begin{theorem}[Chain Rule]
If $g$ is differentiable at $a$ and $f$ is differentiable at $g(a)$, then the composite function $f \circ g$ is differentiable at $a$ and:
\[(f \circ g)'(a) = f'(g(a)) \cdot g'(a)\]
\end{theorem}

\begin{example}[Computing a Derivative]
Let $f(x) = \sin(x^2)$. Using the chain rule:
\[f'(x) = \cos(x^2) \cdot \frac{d}{dx}[x^2] = \cos(x^2) \cdot 2x = 2x\cos(x^2)\]
\end{example}

\subsection{Applications of Derivatives}

\begin{theorem}[Mean Value Theorem]
If $f$ is continuous on $[a, b]$ and differentiable on $(a, b)$, then there exists a number $c \in (a, b)$ such that:
\[f'(c) = \frac{f(b) - f(a)}{b - a}\]
\end{theorem}

\begin{definition}[Critical Point]
A \textbf{critical point} of a function $f$ is a number $c$ in the domain of $f$ where either $f'(c) = 0$ or $f'(c)$ does not exist.
\end{definition}

\section{Integration}

\begin{definition}[Riemann Sum]
For a function $f$ defined on $[a, b]$, a \textbf{Riemann sum} is:
\[S = \sum_{i=1}^{n} f(x_i^*) \Delta x_i\]
where $\Delta x_i = x_i - x_{i-1}$ and $x_i^* \in [x_{i-1}, x_i]$.
\end{definition}

\begin{definition}[Definite Integral]
The \textbf{definite integral} of $f$ from $a$ to $b$ is:
\[\int_a^b f(x) \, dx = \lim_{n \to \infty} \sum_{i=1}^{n} f(x_i^*) \Delta x\]
where $\Delta x = \frac{b-a}{n}$ and $x_i^* = a + i \Delta x$.
\end{definition}

\begin{theorem}[Fundamental Theorem of Calculus]
If $f$ is continuous on $[a, b]$, then:
\begin{enumerate}
    \item If $g(x) = \int_a^x f(t) \, dt$, then $g'(x) = f(x)$
    \item $\int_a^b f(x) \, dx = F(b) - F(a)$ where $F$ is any antiderivative of $f$
\end{enumerate}
\end{theorem}

\begin{example}[Definite Integral]
Evaluate $\int_0^1 x^2 \, dx$:

Since $\frac{d}{dx}\left[\frac{x^3}{3}\right] = x^2$, we have:
\[\int_0^1 x^2 \, dx = \left[\frac{x^3}{3}\right]_0^1 = \frac{1^3}{3} - \frac{0^3}{3} = \frac{1}{3}\]
\end{example}

\section{Techniques of Integration}

\subsection{Integration by Parts}

\begin{theorem}[Integration by Parts]
\[\int u \, dv = uv - \int v \, du\]
\end{theorem}

\begin{example}
Evaluate $\int x e^x \, dx$:

Let $u = x$ and $dv = e^x dx$. Then $du = dx$ and $v = e^x$.
\[\int x e^x \, dx = x e^x - \int e^x \, dx = x e^x - e^x + C = e^x(x - 1) + C\]
\end{example}

\subsection{Trigonometric Substitution}

For integrals involving $\sqrt{a^2 - x^2}$, $\sqrt{a^2 + x^2}$, or $\sqrt{x^2 - a^2}$, we can use trigonometric substitutions:

\begin{itemize}
    \item For $\sqrt{a^2 - x^2}$: use $x = a \sin \theta$
    \item For $\sqrt{a^2 + x^2}$: use $x = a \tan \theta$  
    \item For $\sqrt{x^2 - a^2}$: use $x = a \sec \theta$
\end{itemize}

\section{Series}

\begin{definition}[Infinite Series]
An \textbf{infinite series} is an expression of the form:
\[\sum_{n=1}^{\infty} a_n = a_1 + a_2 + a_3 + \cdots\]
\end{definition}

\begin{definition}[Convergence]
A series $\sum_{n=1}^{\infty} a_n$ \textbf{converges} to $S$ if:
\[\lim_{n \to \infty} S_n = S\]
where $S_n = \sum_{k=1}^{n} a_k$ is the $n$-th partial sum.
\end{definition}

\begin{theorem}[Ratio Test]
For a series $\sum a_n$ with positive terms, let $L = \lim_{n \to \infty} \frac{a_{n+1}}{a_n}$.
\begin{itemize}
    \item If $L < 1$, the series converges
    \item If $L > 1$, the series diverges
    \item If $L = 1$, the test is inconclusive
\end{itemize}
\end{theorem}

\section{Exercises}

\begin{enumerate}
    \item Use the definition of limit to prove that $\lim_{x \to 3} (2x + 1) = 7$.
    
    \item Find the derivative of $f(x) = \ln(\cos(x^2))$.
    
    \item Use the Mean Value Theorem to show that $|\sin a - \sin b| \leq |a - b|$ for all real numbers $a$ and $b$.
    
    \item Evaluate $\int_0^{\pi/2} x \sin x \, dx$ using integration by parts.
    
    \item Determine the convergence of the series $\sum_{n=1}^{\infty} \frac{n!}{n^n}$.
\end{enumerate}
