\chapter{Geometry}

This chapter explores the fundamental concepts of Euclidean geometry, coordinate geometry, and introduces elements of non-Euclidean geometry.

\section{Euclidean Geometry}

\begin{definition}[Point, Line, Plane]
The fundamental objects of Euclidean geometry are:
\begin{itemize}
    \item A \textbf{point} has no dimension (position only)
    \item A \textbf{line} is one-dimensional and extends infinitely in both directions
    \item A \textbf{plane} is two-dimensional and extends infinitely in all directions
\end{itemize}
\end{definition}

\subsection{Euclid's Axioms}

\begin{theorem}[Euclid's Five Postulates]
\begin{enumerate}
    \item A straight line can be drawn between any two points
    \item Any finite straight line can be extended indefinitely
    \item A circle can be drawn with any center and any radius
    \item All right angles are equal to each other
    \item If a line intersects two other lines such that the sum of interior angles on one side is less than two right angles, then the two lines will intersect on that side when extended
\end{enumerate}
\end{theorem}

\begin{theorem}[Parallel Postulate Equivalent]
Through a point not on a given line, there exists exactly one line parallel to the given line.
\end{theorem}

\section{Triangles and Congruence}

\begin{definition}[Triangle]
A \textbf{triangle} is a polygon with three vertices and three sides.
\end{definition}

\begin{theorem}[Triangle Congruence Criteria]
Two triangles are congruent if any of the following conditions hold:
\begin{enumerate}
    \item \textbf{SSS:} Three sides are equal
    \item \textbf{SAS:} Two sides and the included angle are equal
    \item \textbf{ASA:} Two angles and the included side are equal
    \item \textbf{AAS:} Two angles and a non-included side are equal
    \item \textbf{RHS:} Right angle, hypotenuse, and one side are equal
\end{enumerate}
\end{theorem}

\begin{theorem}[Pythagorean Theorem]
In a right triangle with legs of length $a$ and $b$ and hypotenuse of length $c$:
\[a^2 + b^2 = c^2\]
\end{theorem}

\begin{proof}
Consider a square with side length $(a+b)$ containing four copies of the right triangle. The area can be computed in two ways:
\begin{align}
(a+b)^2 &= 4 \cdot \frac{1}{2}ab + c^2 \\
a^2 + 2ab + b^2 &= 2ab + c^2 \\
a^2 + b^2 &= c^2
\end{align}
\end{proof}

\section{Circles}

\begin{definition}[Circle]
A \textbf{circle} is the set of all points in a plane that are equidistant from a fixed point called the center.
\end{definition}

\begin{theorem}[Inscribed Angle Theorem]
An inscribed angle is half the central angle that subtends the same arc.
\end{theorem}

\begin{theorem}[Power of a Point]
For a point $P$ and a circle with center $O$ and radius $r$, the power of $P$ is $|PO|^2 - r^2$. For any line through $P$ intersecting the circle at points $A$ and $B$:
\[PA \cdot PB = ||PO|^2 - r^2|\]
\end{theorem}

\section{Coordinate Geometry}

\begin{definition}[Distance Formula]
The distance between points $(x_1, y_1)$ and $(x_2, y_2)$ is:
\[d = \sqrt{(x_2 - x_1)^2 + (y_2 - y_1)^2}\]
\end{definition}

\begin{definition}[Equation of a Line]
A line can be represented by:
\begin{itemize}
    \item \textbf{Slope-intercept form:} $y = mx + b$
    \item \textbf{Point-slope form:} $y - y_1 = m(x - x_1)$
    \item \textbf{General form:} $Ax + By + C = 0$
\end{itemize}
\end{definition}

\begin{definition}[Equation of a Circle]
A circle with center $(h, k)$ and radius $r$ has equation:
\[(x - h)^2 + (y - k)^2 = r^2\]
\end{definition}

\section{Transformations}

\begin{definition}[Rigid Transformations]
\textbf{Rigid transformations} preserve distances and angles:
\begin{itemize}
    \item \textbf{Translation:} $(x, y) \mapsto (x + a, y + b)$
    \item \textbf{Rotation:} $(x, y) \mapsto (x\cos\theta - y\sin\theta, x\sin\theta + y\cos\theta)$
    \item \textbf{Reflection:} $(x, y) \mapsto (-x, y)$ (across $y$-axis)
\end{itemize}
\end{definition}

\begin{definition}[Similarity Transformations]
\textbf{Similarity transformations} preserve angles but may change distances by a constant factor:
\begin{itemize}
    \item \textbf{Scaling:} $(x, y) \mapsto (kx, ky)$ for some $k > 0$
    \item \textbf{Homothety:} Combination of scaling and translation
\end{itemize}
\end{definition}

\section{Area and Volume}

\begin{theorem}[Area Formulas]
\begin{itemize}
    \item \textbf{Triangle:} $A = \frac{1}{2}bh$ or $A = \sqrt{s(s-a)(s-b)(s-c)}$ (Heron's formula)
    \item \textbf{Rectangle:} $A = lw$
    \item \textbf{Circle:} $A = \pi r^2$
    \item \textbf{Ellipse:} $A = \pi ab$ where $a$ and $b$ are the semi-axes
\end{itemize}
\end{theorem}

\begin{theorem}[Volume Formulas]
\begin{itemize}
    \item \textbf{Rectangular prism:} $V = lwh$
    \item \textbf{Cylinder:} $V = \pi r^2 h$
    \item \textbf{Sphere:} $V = \frac{4}{3}\pi r^3$
    \item \textbf{Cone:} $V = \frac{1}{3}\pi r^2 h$
\end{itemize}
\end{theorem}

\section{Vectors in Geometry}

\begin{definition}[Vector]
A \textbf{vector} is a quantity with both magnitude and direction, often represented as $\overrightarrow{AB}$ or $\mathbf{v} = \langle a, b \rangle$.
\end{definition}

\begin{definition}[Dot Product]
For vectors $\mathbf{u} = \langle u_1, u_2 \rangle$ and $\mathbf{v} = \langle v_1, v_2 \rangle$:
\[\mathbf{u} \cdot \mathbf{v} = u_1 v_1 + u_2 v_2 = |\mathbf{u}||\mathbf{v}|\cos\theta\]
where $\theta$ is the angle between the vectors.
\end{definition}

\begin{definition}[Cross Product]
For vectors $\mathbf{u} = \langle u_1, u_2, u_3 \rangle$ and $\mathbf{v} = \langle v_1, v_2, v_3 \rangle$:
\[\mathbf{u} \times \mathbf{v} = \langle u_2 v_3 - u_3 v_2, u_3 v_1 - u_1 v_3, u_1 v_2 - u_2 v_1 \rangle\]
\end{definition}

\section{Introduction to Non-Euclidean Geometry}

\begin{definition}[Hyperbolic Geometry]
\textbf{Hyperbolic geometry} is a non-Euclidean geometry where the parallel postulate is replaced with: Through a point not on a given line, there exist infinitely many lines parallel to the given line.
\end{definition}

\begin{definition}[Spherical Geometry]
\textbf{Spherical geometry} is the geometry on the surface of a sphere, where "lines" are great circles and there are no parallel lines.
\end{definition}

\begin{theorem}[Gauss-Bonnet Theorem (Simple Form)]
For a triangle on a sphere with angles $\alpha$, $\beta$, and $\gamma$:
\[\alpha + \beta + \gamma = \pi + \frac{A}{R^2}\]
where $A$ is the area of the triangle and $R$ is the radius of the sphere.
\end{theorem}

\section{Exercises}

\begin{enumerate}
    \item Prove that the diagonals of a rhombus are perpendicular bisectors of each other.
    
    \item Find the equation of the circle passing through points $(1, 2)$, $(3, 4)$, and $(5, 2)$.
    
    \item Use vectors to prove that the diagonals of a parallelogram bisect each other.
    
    \item A triangle has sides of length 3, 4, and 5. Find its area using both the base-height formula and Heron's formula.
    
    \item In spherical geometry, what is the sum of angles in a triangle whose area is $\frac{1}{4}$ the area of a hemisphere of radius 1?
\end{enumerate}
