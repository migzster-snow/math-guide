\chapter{Abstract Algebra}

This chapter introduces the fundamental algebraic structures that form the foundation of modern algebra.

\section{Groups}

\begin{definition}[Binary Operation]
A \textbf{binary operation} on a set $S$ is a function $*: S \times S \to S$ that assigns to each ordered pair $(a,b)$ of elements in $S$ an element $a * b$ in $S$.
\end{definition}

\begin{definition}[Group]
A \textbf{group} is a set $G$ together with a binary operation $*$ such that:
\begin{enumerate}
    \item \textbf{Closure:} For all $a, b \in G$, we have $a * b \in G$.
    \item \textbf{Associativity:} For all $a, b, c \in G$, we have $(a * b) * c = a * (b * c)$.
    \item \textbf{Identity:} There exists an element $e \in G$ such that for all $a \in G$, $e * a = a * e = a$.
    \item \textbf{Inverse:} For each $a \in G$, there exists an element $a^{-1} \in G$ such that $a * a^{-1} = a^{-1} * a = e$.
\end{enumerate}
\end{definition}

\begin{example}[The Integers under Addition]
The set $\mathbb{Z}$ with the operation of addition forms a group:
\begin{itemize}
    \item Closure: If $a, b \in \mathbb{Z}$, then $a + b \in \mathbb{Z}$
    \item Associativity: $(a + b) + c = a + (b + c)$ for all $a, b, c \in \mathbb{Z}$
    \item Identity: $0$ is the identity element since $a + 0 = 0 + a = a$
    \item Inverse: For each $a \in \mathbb{Z}$, the inverse is $-a$ since $a + (-a) = 0$
\end{itemize}
\end{example}

\begin{theorem}
In any group $G$, the identity element is unique.
\end{theorem}

\begin{proof}
Suppose $e$ and $e'$ are both identity elements in $G$. Then:
\[e = e * e' = e'\]
where the first equality uses the fact that $e'$ is an identity, and the second equality uses the fact that $e$ is an identity.
\end{proof}

\subsection{Subgroups}

\begin{definition}[Subgroup]
Let $(G, *)$ be a group. A subset $H \subseteq G$ is a \textbf{subgroup} of $G$ if $(H, *)$ is itself a group.
\end{definition}

\begin{theorem}[Subgroup Test]
A non-empty subset $H$ of a group $G$ is a subgroup if and only if:
\begin{enumerate}
    \item For all $a, b \in H$, we have $ab \in H$ (closure)
    \item For all $a \in H$, we have $a^{-1} \in H$ (inverse)
\end{enumerate}
\end{theorem}

\section{Rings}

\begin{definition}[Ring]
A \textbf{ring} is a set $R$ with two binary operations, addition $(+)$ and multiplication $(\cdot)$, such that:
\begin{enumerate}
    \item $(R, +)$ is an abelian group
    \item Multiplication is associative
    \item The distributive laws hold: $a \cdot (b + c) = a \cdot b + a \cdot c$ and $(a + b) \cdot c = a \cdot c + b \cdot c$
\end{enumerate}
\end{definition}

\begin{example}[The Integers]
The set $\mathbb{Z}$ with ordinary addition and multiplication forms a ring.
\end{example}

\begin{definition}[Field]
A \textbf{field} is a commutative ring with unity in which every non-zero element has a multiplicative inverse.
\end{definition}

\begin{example}[The Rational Numbers]
The set $\mathbb{Q}$ of rational numbers forms a field under ordinary addition and multiplication.
\end{example}

\section{Vector Spaces}

\begin{definition}[Vector Space]
Let $F$ be a field. A \textbf{vector space} over $F$ is a set $V$ together with operations of vector addition and scalar multiplication such that:
\begin{enumerate}
    \item $(V, +)$ is an abelian group
    \item For all $\alpha \in F$ and $\mathbf{v} \in V$, we have $\alpha \mathbf{v} \in V$
    \item $\alpha(\mathbf{u} + \mathbf{v}) = \alpha\mathbf{u} + \alpha\mathbf{v}$
    \item $(\alpha + \beta)\mathbf{v} = \alpha\mathbf{v} + \beta\mathbf{v}$
    \item $(\alpha\beta)\mathbf{v} = \alpha(\beta\mathbf{v})$
    \item $1\mathbf{v} = \mathbf{v}$ where $1$ is the multiplicative identity in $F$
\end{enumerate}
\end{definition}

\begin{example}[Euclidean Space]
The set $\mathbb{R}^n$ with componentwise addition and scalar multiplication forms a vector space over $\mathbb{R}$.
\end{example}

\begin{definition}[Linear Independence]
Vectors $\mathbf{v_1}, \mathbf{v_2}, \ldots, \mathbf{v_n}$ in a vector space $V$ are \textbf{linearly independent} if the only solution to
\[\alpha_1 \mathbf{v_1} + \alpha_2 \mathbf{v_2} + \cdots + \alpha_n \mathbf{v_n} = \mathbf{0}\]
is $\alpha_1 = \alpha_2 = \cdots = \alpha_n = 0$.
\end{definition}

\begin{definition}[Basis]
A \textbf{basis} for a vector space $V$ is a set of vectors that is both linearly independent and spans $V$.
\end{definition}

\section{Exercises}

\begin{enumerate}
    \item Prove that in any group, each element has a unique inverse.
    
    \item Show that the set of even integers forms a subgroup of $(\mathbb{Z}, +)$.
    
    \item Verify that the set $\mathbb{Z}/n\mathbb{Z}$ forms a ring under addition and multiplication modulo $n$.
    
    \item Prove that if $\mathbf{v_1}, \mathbf{v_2}, \ldots, \mathbf{v_n}$ are linearly independent, then any subset is also linearly independent.
    
    \item Find a basis for the vector space of $2 \times 2$ matrices over $\mathbb{R}$.
\end{enumerate}
