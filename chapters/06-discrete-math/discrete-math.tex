\chapter{Discrete Mathematics}

Discrete mathematics studies mathematical structures that are fundamentally discrete rather than continuous, forming the foundation for computer science and combinatorics.

\section{Graph Theory}

\begin{definition}[Graph]
A \textbf{graph} $G = (V, E)$ consists of a finite set $V$ of vertices (or nodes) and a set $E$ of edges, where each edge connects two vertices.
\end{definition}

\begin{definition}[Degree]
The \textbf{degree} of a vertex $v$, denoted $\deg(v)$, is the number of edges incident to $v$.
\end{definition}

\begin{theorem}[Handshaking Lemma]
For any graph $G = (V, E)$:
\[\sum_{v \in V} \deg(v) = 2|E|\]
\end{theorem}

\begin{proof}
Each edge contributes 1 to the degree of each of its two endpoints, so the sum of all degrees counts each edge exactly twice.
\end{proof}

\begin{definition}[Path]
A \textbf{path} in a graph is a sequence of vertices $v_1, v_2, \ldots, v_k$ such that for each $i$, there is an edge between $v_i$ and $v_{i+1}$.
\end{definition}

\begin{definition}[Connected Graph]
A graph is \textbf{connected} if there is a path between every pair of vertices.
\end{definition}

\begin{definition}[Tree]
A \textbf{tree} is a connected graph with no cycles.
\end{definition}

\begin{theorem}[Tree Characterization]
For a graph $G$ with $n$ vertices, the following are equivalent:
\begin{enumerate}
    \item $G$ is a tree
    \item $G$ is connected and has $n-1$ edges
    \item $G$ is acyclic and has $n-1$ edges
    \item There is exactly one path between any two vertices in $G$
\end{enumerate}
\end{theorem}

\section{Combinatorics}

\begin{definition}[Permutation]
A \textbf{permutation} of $n$ distinct objects is an arrangement of these objects in a specific order. The number of permutations of $n$ objects is $n!$.
\end{definition}

\begin{definition}[Combination]
A \textbf{combination} is a selection of objects where order does not matter. The number of ways to choose $k$ objects from $n$ objects is:
\[\binom{n}{k} = \frac{n!}{k!(n-k)!}\]
\end{definition}

\begin{theorem}[Binomial Theorem]
For any real numbers $x$ and $y$ and positive integer $n$:
\[(x + y)^n = \sum_{k=0}^{n} \binom{n}{k} x^{n-k} y^k\]
\end{theorem}

\begin{example}[Counting Problem]
How many ways can we arrange the letters in "MATHEMATICS"?

The word has 11 letters with repetitions: M(2), A(2), T(2), H(1), E(1), I(1), C(1), S(1).

The number of arrangements is:
\[\frac{11!}{2! \cdot 2! \cdot 2! \cdot 1! \cdot 1! \cdot 1! \cdot 1! \cdot 1!} = \frac{11!}{2^3} = \frac{39,916,800}{8} = 4,989,600\]
\end{example}

\subsection{Inclusion-Exclusion Principle}

\begin{theorem}[Inclusion-Exclusion Principle]
For finite sets $A_1, A_2, \ldots, A_n$:
\begin{align}
|A_1 \cup A_2 \cup \cdots \cup A_n| &= \sum_{i} |A_i| - \sum_{i < j} |A_i \cap A_j| \\
&\quad + \sum_{i < j < k} |A_i \cap A_j \cap A_k| - \cdots + (-1)^{n+1}|A_1 \cap A_2 \cap \cdots \cap A_n|
\end{align}
\end{theorem}

\section{Number Theory}

\begin{definition}[Divisibility]
An integer $a$ \textbf{divides} an integer $b$ (written $a \mid b$) if there exists an integer $k$ such that $b = ak$.
\end{definition}

\begin{theorem}[Division Algorithm]
For any integers $a$ and $b$ with $b > 0$, there exist unique integers $q$ and $r$ such that:
\[a = bq + r \quad \text{where } 0 \leq r < b\]
\end{theorem}

\begin{definition}[Greatest Common Divisor]
The \textbf{greatest common divisor} of integers $a$ and $b$, denoted $\gcd(a, b)$, is the largest positive integer that divides both $a$ and $b$.
\end{definition}

\begin{theorem}[Euclidean Algorithm]
The Euclidean algorithm computes $\gcd(a, b)$ by repeatedly applying:
\[\gcd(a, b) = \gcd(b, a \bmod b)\]
until one of the numbers becomes 0.
\end{theorem}

\begin{theorem}[Bézout's Identity]
For any integers $a$ and $b$, there exist integers $x$ and $y$ such that:
\[\gcd(a, b) = ax + by\]
\end{theorem}

\subsection{Prime Numbers}

\begin{definition}[Prime Number]
A positive integer $p > 1$ is \textbf{prime} if its only positive divisors are 1 and $p$.
\end{definition}

\begin{theorem}[Fundamental Theorem of Arithmetic]
Every integer greater than 1 can be expressed uniquely (up to order) as a product of prime numbers.
\end{theorem}

\begin{theorem}[Infinitude of Primes]
There are infinitely many prime numbers.
\end{theorem}

\begin{proof}[Euclid's Proof]
Suppose there are only finitely many primes $p_1, p_2, \ldots, p_k$. Consider:
\[N = p_1 p_2 \cdots p_k + 1\]

Since $N > 1$, it has a prime divisor $p$. But $p$ cannot be any of $p_1, \ldots, p_k$ since $N \equiv 1 \pmod{p_i}$ for each $i$. This contradicts our assumption that we listed all primes.
\end{proof}

\section{Recurrence Relations}

\begin{definition}[Recurrence Relation]
A \textbf{recurrence relation} is an equation that defines a sequence recursively, expressing each term in terms of previous terms.
\end{definition}

\begin{example}[Fibonacci Sequence]
The Fibonacci sequence is defined by:
\[F_0 = 0, \quad F_1 = 1, \quad F_n = F_{n-1} + F_{n-2} \text{ for } n \geq 2\]
\end{example}

\begin{theorem}[Linear Homogeneous Recurrence Relations]
The recurrence relation $a_n = c_1 a_{n-1} + c_2 a_{n-2}$ with characteristic equation $r^2 - c_1 r - c_2 = 0$ has:
\begin{itemize}
    \item If roots $r_1 \neq r_2$: general solution $a_n = A r_1^n + B r_2^n$
    \item If repeated root $r$: general solution $a_n = (A + Bn) r^n$
\end{itemize}
\end{theorem}

\section{Boolean Algebra}

\begin{definition}[Boolean Algebra]
A \textbf{Boolean algebra} is a set $B$ with two binary operations $\land$ (AND) and $\lor$ (OR), a unary operation $\neg$ (NOT), and constants 0 and 1, satisfying certain axioms.
\end{definition}

\begin{theorem}[De Morgan's Laws]
For any Boolean expressions $A$ and $B$:
\begin{align}
\neg(A \land B) &= \neg A \lor \neg B \\
\neg(A \lor B) &= \neg A \land \neg B
\end{align}
\end{theorem}

\begin{definition}[Boolean Function]
A \textbf{Boolean function} is a function $f: \{0,1\}^n \to \{0,1\}$ that maps $n$-tuples of Boolean values to a Boolean value.
\end{definition}

\section{Exercises}

\begin{enumerate}
    \item Prove that in any graph, the number of vertices with odd degree is even.
    
    \item Find the number of ways to distribute 10 identical balls into 4 distinct boxes.
    
    \item Use the inclusion-exclusion principle to find the number of integers from 1 to 100 that are divisible by 2, 3, or 5.
    
    \item Find $\gcd(252, 198)$ using the Euclidean algorithm and express it in the form $252x + 198y$.
    
    \item Solve the recurrence relation $a_n = 5a_{n-1} - 6a_{n-2}$ with initial conditions $a_0 = 1, a_1 = 0$.
\end{enumerate}