\chapter{Mathematical Foundations}

This chapter establishes the fundamental concepts and notation that will be used throughout the textbook. We begin with set theory, logic, and proof techniques.

\section{Set Theory}

\begin{definition}[Set]
A \textbf{set} is a well-defined collection of distinct objects, called elements or members of the set.
\end{definition}

\begin{example}
The set of natural numbers less than 5 can be written as:
\[N_5 = \{1, 2, 3, 4\}\]
\end{example}

\subsection{Set Operations}

Let $A$ and $B$ be sets. We define the following operations:

\begin{definition}[Union]
The \textbf{union} of sets $A$ and $B$ is:
\[A \cup B = \{x : x \in A \text{ or } x \in B\}\]
\end{definition}

\begin{definition}[Intersection]
The \textbf{intersection} of sets $A$ and $B$ is:
\[A \cap B = \{x : x \in A \text{ and } x \in B\}\]
\end{definition}

\begin{definition}[Complement]
The \textbf{complement} of set $A$ with respect to universal set $U$ is:
\[A^c = \{x \in U : x \notin A\}\]
\end{definition}

\section{Logic and Proof Techniques}

\subsection{Logical Statements}

\begin{definition}[Proposition]
A \textbf{proposition} is a statement that is either true or false, but not both.
\end{definition}

\begin{example}
The following are propositions:
\begin{itemize}
    \item $2 + 2 = 4$ (True)
    \item All prime numbers are odd (False, since 2 is prime and even)
    \item $\sqrt{2}$ is irrational (True)
\end{itemize}
\end{example}

\subsection{Methods of Proof}

\begin{definition}[Direct Proof]
A \textbf{direct proof} of a statement "If $P$, then $Q$" assumes $P$ is true and uses logical steps to show that $Q$ must be true.
\end{definition}

\begin{theorem}
If $n$ is an even integer, then $n^2$ is even.
\end{theorem}

\begin{proof}
Let $n$ be an even integer. Then $n = 2k$ for some integer $k$. We have:
\[n^2 = (2k)^2 = 4k^2 = 2(2k^2)\]
Since $2k^2$ is an integer, $n^2 = 2(2k^2)$ is even.
\end{proof}

\begin{definition}[Proof by Contradiction]
A \textbf{proof by contradiction} assumes the negation of what we want to prove and shows this leads to a logical contradiction.
\end{definition}

\begin{theorem}
$\sqrt{2}$ is irrational.
\end{theorem}

\begin{proof}
Assume for contradiction that $\sqrt{2}$ is rational. Then $\sqrt{2} = \frac{p}{q}$ where $p$ and $q$ are integers with $\gcd(p,q) = 1$.

Squaring both sides: $2 = \frac{p^2}{q^2}$, so $2q^2 = p^2$.

This means $p^2$ is even, which implies $p$ is even. Let $p = 2r$ for some integer $r$.

Substituting: $2q^2 = (2r)^2 = 4r^2$, so $q^2 = 2r^2$.

This means $q^2$ is even, which implies $q$ is even.

But if both $p$ and $q$ are even, then $\gcd(p,q) \geq 2$, contradicting our assumption that $\gcd(p,q) = 1$.

Therefore, $\sqrt{2}$ is irrational.
\end{proof}

\section{Functions and Relations}

\begin{definition}[Function]
A \textbf{function} $f: A \to B$ is a relation that assigns to each element $a \in A$ exactly one element $f(a) \in B$.
\end{definition}

\begin{definition}[Injective Function]
A function $f: A \to B$ is \textbf{injective} (one-to-one) if for all $a_1, a_2 \in A$:
\[f(a_1) = f(a_2) \implies a_1 = a_2\]
\end{definition}

\begin{definition}[Surjective Function]
A function $f: A \to B$ is \textbf{surjective} (onto) if for all $b \in B$, there exists $a \in A$ such that $f(a) = b$.
\end{definition}

\begin{definition}[Bijective Function]
A function is \textbf{bijective} if it is both injective and surjective.
\end{definition}

\section{Exercises}

\begin{enumerate}
    \item Prove that the intersection of two sets is commutative: $A \cap B = B \cap A$.
    
    \item Show that if $f: A \to B$ and $g: B \to C$ are both injective, then $g \circ f: A \to C$ is injective.
    
    \item Prove by contradiction that there are infinitely many prime numbers.
    
    \item Let $A = \{1, 2, 3, 4\}$ and $B = \{2, 4, 6, 8\}$. Find $A \cup B$, $A \cap B$, and $A \setminus B$.
\end{enumerate}
