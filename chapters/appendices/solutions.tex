\chapter{Selected Solutions}

This appendix provides detailed solutions to selected exercises from each chapter.

\section{Chapter 1: Mathematical Foundations}

\subsection{Exercise 1}
\textbf{Problem:} Prove that the intersection of two sets is commutative: $A \cap B = B \cap A$.

\textbf{Solution:}
We need to show that $A \cap B = B \cap A$ by proving two inclusions.

First, we show $A \cap B \subseteq B \cap A$:
Let $x \in A \cap B$. By definition of intersection, $x \in A$ and $x \in B$.
Since $x \in B$ and $x \in A$, we have $x \in B \cap A$.
Therefore, $A \cap B \subseteq B \cap A$.

Next, we show $B \cap A \subseteq A \cap B$:
Let $x \in B \cap A$. By definition of intersection, $x \in B$ and $x \in A$.
Since $x \in A$ and $x \in B$, we have $x \in A \cap B$.
Therefore, $B \cap A \subseteq A \cap B$.

Since both inclusions hold, $A \cap B = B \cap A$.

\subsection{Exercise 3}
\textbf{Problem:} Prove by contradiction that there are infinitely many prime numbers.

\textbf{Solution:}
Assume for contradiction that there are only finitely many prime numbers. Let these primes be $p_1, p_2, \ldots, p_k$.

Consider the number $N = p_1 \cdot p_2 \cdot \ldots \cdot p_k + 1$.

Since $N > 1$, by the fundamental theorem of arithmetic, $N$ must have at least one prime divisor. Let $p$ be a prime divisor of $N$.

If $p$ is one of $p_1, p_2, \ldots, p_k$, then $p$ divides the product $p_1 \cdot p_2 \cdot \ldots \cdot p_k$. Since $p$ also divides $N$, it must divide their difference:
$N - p_1 \cdot p_2 \cdot \ldots \cdot p_k = 1$

But no prime can divide 1, which is a contradiction.

Therefore, $p$ cannot be any of $p_1, p_2, \ldots, p_k$, meaning we have found a new prime not in our original list. This contradicts our assumption that we had listed all primes.

Therefore, there must be infinitely many prime numbers.

\section{Chapter 2: Abstract Algebra}

\subsection{Exercise 1}
\textbf{Problem:} Prove that in any group, each element has a unique inverse.

\textbf{Solution:}
Let $G$ be a group with operation $*$ and identity element $e$. Let $a \in G$.

Suppose $b$ and $c$ are both inverses of $a$. Then:
- $a * b = b * a = e$
- $a * c = c * a = e$

We need to show $b = c$.

Starting with $b$:
\begin{align}
b &= b * e \quad \text{(identity property)} \\
&= b * (a * c) \quad \text{(since $a * c = e$)} \\
&= (b * a) * c \quad \text{(associativity)} \\
&= e * c \quad \text{(since $b * a = e$)} \\
&= c \quad \text{(identity property)}
\end{align}

Therefore, $b = c$, proving uniqueness of the inverse.

\section{Chapter 3: Calculus}

\subsection{Exercise 1}
\textbf{Problem:} Use the definition of limit to prove that $\lim_{x \to 3} (2x + 1) = 7$.

\textbf{Solution:}
We need to show that for every $\varepsilon > 0$, there exists $\delta > 0$ such that whenever $0 < |x - 3| < \delta$, we have $|(2x + 1) - 7| < \varepsilon$.

First, let's simplify the expression we need to bound:
$|(2x + 1) - 7| = |2x - 6| = 2|x - 3|$

For this to be less than $\varepsilon$, we need:
$2|x - 3| < \varepsilon$
$|x - 3| < \frac{\varepsilon}{2}$

Therefore, we can choose $\delta = \frac{\varepsilon}{2}$.

Verification: If $0 < |x - 3| < \delta = \frac{\varepsilon}{2}$, then:
$|(2x + 1) - 7| = 2|x - 3| < 2 \cdot \frac{\varepsilon}{2} = \varepsilon$

This proves that $\lim_{x \to 3} (2x + 1) = 7$.

\subsection{Exercise 4}
\textbf{Problem:} Evaluate $\int_0^{\pi/2} x \sin x \, dx$ using integration by parts.

\textbf{Solution:}
Using integration by parts with $u = x$ and $dv = \sin x \, dx$:
- $du = dx$
- $v = -\cos x$

Applying the integration by parts formula:
\begin{align}
\int_0^{\pi/2} x \sin x \, dx &= \left[x(-\cos x)\right]_0^{\pi/2} - \int_0^{\pi/2} (-\cos x) \, dx \\
&= \left[-x \cos x\right]_0^{\pi/2} + \int_0^{\pi/2} \cos x \, dx \\
&= \left[-x \cos x\right]_0^{\pi/2} + \left[\sin x\right]_0^{\pi/2} \\
&= \left[-\frac{\pi}{2} \cos\frac{\pi}{2} - (-0 \cos 0)\right] + \left[\sin\frac{\pi}{2} - \sin 0\right] \\
&= [0 - 0] + [1 - 0] \\
&= 1
\end{align}

Therefore, $\int_0^{\pi/2} x \sin x \, dx = 1$.

\section{Chapter 4: Linear Algebra}

\subsection{Exercise 2}
\textbf{Problem:} Find the determinant of $A = \begin{pmatrix} 2 & -1 & 3 \\ 1 & 0 & 4 \\ -2 & 1 & 1 \end{pmatrix}$.

\textbf{Solution:}
Using cofactor expansion along the second row (which has a zero):

$\det(A) = 1 \cdot \det\begin{pmatrix} -1 & 3 \\ 1 & 1 \end{pmatrix} - 0 \cdot (\text{something}) + 4 \cdot \det\begin{pmatrix} 2 & -1 \\ -2 & 1 \end{pmatrix}$

Computing the 2×2 determinants:
- $\det\begin{pmatrix} -1 & 3 \\ 1 & 1 \end{pmatrix} = (-1)(1) - (3)(1) = -1 - 3 = -4$
- $\det\begin{pmatrix} 2 & -1 \\ -2 & 1 \end{pmatrix} = (2)(1) - (-1)(-2) = 2 - 2 = 0$

Therefore:
$\det(A) = 1 \cdot (-4) - 0 + 4 \cdot 0 = -4$

\section{Chapter 5: Real Analysis}

\subsection{Exercise 1}
\textbf{Problem:} Prove that $\lim_{n \to \infty} \frac{1}{n} = 0$ using the definition of convergence.

\textbf{Solution:}
We need to show that for every $\varepsilon > 0$, there exists $N \in \mathbb{N}$ such that for all $n \geq N$:
$\left|\frac{1}{n} - 0\right| < \varepsilon$

This simplifies to showing $\frac{1}{n} < \varepsilon$.

Given $\varepsilon > 0$, by the Archimedean property, there exists a positive integer $N$ such that $N > \frac{1}{\varepsilon}$, which means $\frac{1}{N} < \varepsilon$.

For any $n \geq N$, we have $n \geq N > \frac{1}{\varepsilon}$, so $\frac{1}{n} \leq \frac{1}{N} < \varepsilon$.

Therefore, $\left|\frac{1}{n} - 0\right| = \frac{1}{n} < \varepsilon$ for all $n \geq N$.

This proves that $\lim_{n \to \infty} \frac{1}{n} = 0$.

\section{Chapter 6: Discrete Mathematics}

\subsection{Exercise 1}
\textbf{Problem:} Prove that in any graph, the number of vertices with odd degree is even.

\textbf{Solution:}
Let $G = (V, E)$ be a graph. Let $V_{\text{odd}}$ be the set of vertices with odd degree and $V_{\text{even}}$ be the set of vertices with even degree.

By the handshaking lemma:
$\sum_{v \in V} \deg(v) = 2|E|$

Since $2|E|$ is even, the sum of all degrees is even.

We can partition this sum:
$\sum_{v \in V} \deg(v) = \sum_{v \in V_{\text{even}}} \deg(v) + \sum_{v \in V_{\text{odd}}} \deg(v)$

The first sum, $\sum_{v \in V_{\text{even}}} \deg(v)$, is a sum of even numbers, so it's even.

Since the total sum is even and the first part is even, the second part $\sum_{v \in V_{\text{odd}}} \deg(v)$ must also be even.

But this is a sum of odd numbers. For a sum of odd numbers to be even, there must be an even number of terms.

Therefore, $|V_{\text{odd}}|$ is even, meaning the number of vertices with odd degree is even.

\section{Chapter 7: Probability Theory}

\subsection{Exercise 2}
\textbf{Problem:} If $X \sim \text{Binomial}(10, 0.3)$, compute $P(X = 3)$ and $E[X]$.

\textbf{Solution:}
For a binomial distribution with parameters $n = 10$ and $p = 0.3$:

$P(X = 3) = \binom{10}{3} (0.3)^3 (0.7)^7$

Computing each part:
- $\binom{10}{3} = \frac{10!}{3! \cdot 7!} = \frac{10 \cdot 9 \cdot 8}{3 \cdot 2 \cdot 1} = 120$
- $(0.3)^3 = 0.027$
- $(0.7)^7 = 0.0823543$

Therefore:
$P(X = 3) = 120 \times 0.027 \times 0.0823543 \approx 0.2668$

For the expected value:
$E[X] = np = 10 \times 0.3 = 3$

\section{Chapter 8: Geometry}

\subsection{Exercise 4}
\textbf{Problem:} A triangle has sides of length 3, 4, and 5. Find its area using both the base-height formula and Heron's formula.

\textbf{Solution:}
First, note that $3^2 + 4^2 = 9 + 16 = 25 = 5^2$, so this is a right triangle with legs of length 3 and 4, and hypotenuse of length 5.

\textbf{Method 1: Base-height formula}
Using the legs as base and height:
$A = \frac{1}{2} \times \text{base} \times \text{height} = \frac{1}{2} \times 3 \times 4 = 6$

\textbf{Method 2: Heron's formula}
First, find the semi-perimeter:
$s = \frac{3 + 4 + 5}{2} = 6$

Then apply Heron's formula:
\begin{align}
A &= \sqrt{s(s-a)(s-b)(s-c)} \\
&= \sqrt{6(6-3)(6-4)(6-5)} \\
&= \sqrt{6 \times 3 \times 2 \times 1} \\
&= \sqrt{36} \\
&= 6
\end{align}

Both methods give the same result: the area is 6 square units.
