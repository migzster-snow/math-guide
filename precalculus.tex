\part{Precalculus}

\section{Logical Notation}
\begin{description}
    \item[for all] $\forall$
    \item[there exists] $\exists$
    \item[there does not exist] $\nexists$
    \item[implies] $\implies$
    \item[if and only if] $\iff$
    \item[equivalent to] $\equiv$
    \item[not equivalent to] $\not\equiv$
\end{description}

\section{Real Number Systems}
\begin{description}
    \item[Natural Numbers] $\mathbb{N} = \{1,2,3,\dots\}$
    \item[Whole Numbers] $\mathbb{W} = \{0,1,2,3,\dots\}$
    \item[Integers] $\mathbb{Z} = \{0,\pm1,\pm2,\pm3,\dots\}$
    \item[Rational Numbers] $\mathbb{Q} = \left\{\frac{a}{b} : a,b \in \mathbb{Z},\, b \neq 0 \right\}$
    \item[Irrational Numbers] $\mathbb{I} = \mathbb{R} \setminus \mathbb{Q}$
    \item[Real Numbers] $\mathbb{R} = \mathbb{Q} \cup \mathbb{I}$
\end{description}

$\mathbb{N} \subset \mathbb{W} \subset \mathbb{Z} \subset \mathbb{Q} \subset \mathbb{R}$

\section{Properties of Real Numbers}

\subsection{Commutative Property}
\begin{description}
    \item[addition] $a + b = b + a$
    \item[multiplication] $ab = ba$
\end{description}

\subsection{Associative Property}
\begin{description}
    \item[addition] $(a + b) + c = a + (b + c)$
    \item[multiplication] $(ab)c = a(bc)$
\end{description}

\subsection{Distributive Property}
\begin{description}
    \item[distributive] $a(b + c) = ab + ac$
    \item[distributive] $(b + c)a = ba + ca$
\end{description}

\section{Properties of Negatives}
\begin{enumerate}
    \item $(-1)a = -a$
    \item $-(-a) = a$
    \item $(-a)b = a(-b) = -(ab)$
    \item $(-a)(-b) = ab$
    \item $-(a + b) = -a + -b$
    \item $-(a - b) = -a + b$
\end{enumerate}

\section{Properties of Fractions}
\begin{enumerate}
    \item $\frac{a}{b} \cdot \frac{c}{d} = \frac{ac}{bd}$
    \item $\frac{a}{b} \div \frac{c}{d} = \frac{a}{b} \cdot \frac{d}{c}$
    \item $\frac{a}{c} + \frac{b}{c} = \frac{a + b}{c}$
    \item $\frac{a}{b} + \frac{c}{d} = \frac{ad + bc}{bd}$
    \item $\frac{ac}{bc} = \frac{a}{b}$
    \item If $\frac{a}{b} = \frac{c}{d}$, then $ad = bc$
\end{enumerate}

\section{The Real Line}

The real number line is ordered. For any two real numbers $a$ and $b$, one and only one of the following is true:
\begin{itemize}
    \item $a < b$
    \item $a = b$
    \item $a > b$
\end{itemize}

\section{Equality/Inequality Notation}
\begin{description}
    \item[less than] $<$
    \item[greater than] $>$
    \item[less than or equal to] $\leq$
    \item[greater than or equal to] $\geq$
    \item[equal to] $=$
    \item[not equal to] $\neq$
\end{description}

\section{Set Notation}
\begin{description}
    \item[set of elements] $\{\}$
    \item[such that] $\mid$ or $:$
    \item[is an element of] $\in$
    \item[is not an element of] $\notin$
    \item[is a subset of] $\subseteq$
    \item[is a proper subset of] $\subset$
    \item[is a superset of] $\supseteq$
    \item[is a proper superset of] $\supset$
    \item[empty set] $\emptyset$
    \item[universal set] $U$
    \item[cardinality] $|A|$
\end{description}

\section{Set Operations}
\begin{description}
    \item[union] $\cup$
    \item[intersection] $\cap$
    \item[difference] $\setminus$
    \item[complement] $A'$
\end{description}

\section{Interval Notation}
\begin{description}
    \item[open interval] $(a, b) = \{x \in \mathbb{R} : a < x < b\}$
    \item[closed interval] $[a, b] = \{x \in \mathbb{R} : a \leq x \leq b\}$
    \item[half-open interval] $[a, b) = \{x \in \mathbb{R} : a \leq x < b\}$
    \item[half-open interval] $(a, b] = \{x \in \mathbb{R} : a < x \leq b\}$
    \item[infinite interval] $(a, \infty) = \{x \in \mathbb{R} : x > a\}$
    \item[infinite interval] $[a, \infty) = \{x \in \mathbb{R} : x \geq a\}$
    \item[infinite interval] $(-\infty, b) = \{x \in \mathbb{R} : x < b\}$
    \item[infinite interval] $(-\infty, b] = \{x \in \mathbb{R} : x \leq b\}$
    \item[infinite interval] $(-\infty, \infty) = \mathbb{R}$
\end{description}

\section{Absolute Value and Distance}
\begin{description}
    \item[absolute value] $|a| =
        \begin{cases}
            a, & \text{if } a \geq 0 \\
            -a, & \text{if } a < 0
        \end{cases}$
    \item[distance between two points] $d(a, b) = |a - b|$
\end{description}

\subsection{Properties of Absolute Value}
\begin{enumerate}
    \item $|ab| = |a||b|$
    \item \textbf{positive square root} $|a|^2 = a^2 \implies |a| = \sqrt{a^2}$ 
    \item $|a^n| = |a|^n$, where $n \in \mathbb{Z}$ and $a \neq 0$ for $n < 0$
    \item \textbf{triangle inequality} $|a + b| \leq |a| + |b|$
\end{enumerate}

\section{Questions}

Precalculus - Mathematics for Calculus by James Stewart, Section 1.1.

\section{Exponential Notation}
\begin{description}
    \item $a^n = a \cdot a \cdot \ldots \cdot a$ ($a$ multiplied $n$ times)
\end{description}

\subsection{Zero and Negative Exponents}
\begin{description}
    \item $a^0 = 1$, where $a \neq 0$
    \item $a^{-n} = \frac{1}{a^n}$,
\end{description}

\subsection{Laws of Exponents}
\begin{enumerate}
    \item $a^m \cdot a^n = a^{m+n}$
    \item $\frac{a^m}{a^n} = a^{m-n}$
    \item $(a^m)^n = a^{mn}$
    \item $(ab)^n = a^n b^n$
    \item $\left(\frac{a}{b}\right)^n = \frac{a^n}{b^n}$
    \item $\left(\frac{a}{b}\right)^{-n} = \left(\frac{b}{a}\right)^n$
    \item $\frac{a^{-n}}{b^{-m}} = \frac{b^m}{a^n}$
\end{enumerate}

\section{Scientific Notation}
\begin{description}
    \item[scientific notation] $x = a \times 10^n$, where $1 \leq |a| < 10$ and $n \in \mathbb{Z}$ 
\end{description}

\section{Radicals}
\begin{description}
    \item[radical] $\sqrt{a} = b \iff b^2 = a$ and $b \geq 0$
    \item[nth root] $\sqrt[n]{a} = b \iff b^n = a$ and if $n$ is even, then $a, b \geq 0$ 
\end{description}

\subsection{Properties of nth Roots}
\begin{enumerate}
    \item $\sqrt[n]{ab} = \sqrt[n]{a} \cdot \sqrt[n]{b}$
    \item $\sqrt[n]{\frac{a}{b}} = \frac{\sqrt[n]{a}}{\sqrt[n]{b}}$
    \item $\sqrt[m]{\sqrt[n]{a}} = \sqrt[mn]{a}$
    \item $\sqrt[n]{a^n} = a$ if $n$ is odd
    \item $\sqrt[n]{a^n} = |a|$ if $n$ is even
\end{enumerate}

\section{Rational Exponents}
\begin{description}
    \item[rational exponent] $a^{\frac{1}{n}} = \sqrt[n]{a}$ and $a^{\frac{m}{n}} = \left(\sqrt[n]{a}\right)^m = \sqrt[n]{a^m}$
\end{description}

\section{Questions}

Precalculus - Mathematics for Calculus by James Stewart, Section 1.2.