\documentclass[12pt, letterpaper]{book}
\usepackage{graphicx}
\usepackage{amsfonts}
\usepackage{amsmath}

\title{Math Guide}
\author{Jonah Benedicto\thanks{The University of Queensland}}
\date{September 2025}

\begin{document}
\maketitle

\tableofcontents

\newpage

\section*{Preface}
This is a math guide for The University of Queensland's MATH1051, MATH1052, MATH2001, MATH1061, and MATH2302 courses. It is intended to be a quick reference for students studying these subjects.

\part{MATH1051 Calculus \& Linear Algebra I}

\chapter{Eigenvalues and Eigenvectors}

Consider a linear transformation represented by a square matrix, denoted $A$. When a vector is transformed by the matrix, its direction and magnitude can change. \textbf{Eigenvectors}, denoted $\underline{x}$, are special vectors that only change in magnitude (not direction) under this transformation. The magnitude by which the eigenvector is scaled is called the \textbf{eigenvalue}, denoted $\lambda$.

The relationship between a matrix $A$, its eigenvalues $\lambda$, and eigenvectors $\underline{x}$ is given by the equation:

$$A \underline{x} = \lambda \underline{x}$$

This is known as the \textbf{definition of eigenvalues and eigenvectors}.

\section{Eigenvalue Equation}

To find the eigenvalues $\lambda$ of a matrix $A$, we solve the equation:
$$|A - \lambda I| = 0$$

\section{Eigenvector Equation}

To find the eigenvectors $\underline{x}$ corresponding to a specific eigenvalue $\lambda$, we solve the equation:
$$(A - \lambda I) \underline{x} = \underline{0}$$

\section{Derivation of Eigenvalue and Eigenvector Equations}

To find the eigenvector equation, we can rearrange the definition of eigenvalues and eigenvectors. 

$$
\begin{aligned}
A \underline{x} &= \lambda \underline{x} \\
\iff A \underline{x} &= \lambda I \underline{x} \\
\iff A \underline{x} - \lambda \underline{x} &= \underline{0} \\
\iff (A - \lambda I) \underline{x} &= \underline{0} \\
\end{aligned}
$$

To find the eigenvalue equation, note that eigenvectors are non-zero vectors. Therefore, the equation $(A - \lambda I) \underline{x} = \underline{0}$ must have non-trivial solutions. This occurs when the matrix $(A - \lambda I)$ is singular, when its determinant is zero:
$$|A - \lambda I| = 0$$

Therefore, we have derived both the eigenvalue and eigenvector equations from the definition of eigenvalues and eigenvectors.

\part{MATH1052 Multivariable Calculus \& Linear Algebra I}

\part{MATH2001 Calculus \& Linear Algebra II}

\part{MATH1061 Discrete Mathematics I}

\part{MATH2302 Discrete Mathematics II}

\end{document}