\documentclass[12pt,letterpaper]{book}

% Packages
\usepackage[utf8]{inputenc}
\usepackage[T1]{fontenc}
\usepackage{amsmath}
\usepackage{amsfonts}
\usepackage{amssymb}
\usepackage{amsthm}
\usepackage{mathtools}
\usepackage{geometry}
\usepackage{fancyhdr}
\usepackage{graphicx}
\usepackage{hyperref}
\usepackage{xcolor}
\usepackage{tikz}
\usepackage{pgfplots}
\usepackage{enumitem}
\usepackage{tcolorbox}

% Fix pgfplots compatibility
\pgfplotsset{compat=1.18}

% Page geometry
\geometry{margin=1in}

% Fix header height for fancyhdr
\setlength{\headheight}{14.49998pt}
\addtolength{\topmargin}{-2.49998pt}

% Header and footer
\pagestyle{fancy}
\fancyhf{}
\fancyhead[LE,RO]{\thepage}
\fancyhead[LO]{\rightmark}
\fancyhead[RE]{\leftmark}

% Theorem environments
\theoremstyle{definition}
\newtheorem{definition}{Definition}[chapter]
\newtheorem{theorem}{Theorem}[chapter]
\newtheorem{lemma}{Lemma}[chapter]
\newtheorem{corollary}{Corollary}[chapter]
\newtheorem{proposition}{Proposition}[chapter]
\newtheorem{example}{Example}[chapter]
\newtheorem{remark}{Remark}[chapter]

% Custom colors - formal and professional
\definecolor{theoremcolor}{RGB}{47,79,79}      % Dark slate gray
\definecolor{definitioncolor}{RGB}{25,25,112}  % Midnight blue
\definecolor{examplecolor}{RGB}{105,105,105}   % Dim gray
\definecolor{remarkcolor}{RGB}{139,69,19}      % Saddle brown

% Define common style for boxes
\tcbset{
  commonbox/.style={
    boxrule=0.8pt,
    arc=2pt,
    left=10pt,
    right=10pt,
    top=8pt,
    bottom=8pt,
    fonttitle=\bfseries
  }
}

% Colored boxes for theorems - formal styling
\tcolorboxenvironment{theorem}{
  commonbox,
  colback=theoremcolor!8!white,
  colframe=theoremcolor!80!black,
  title=Theorem
}

\tcolorboxenvironment{definition}{
  commonbox,
  colback=definitioncolor!6!white,
  colframe=definitioncolor!70!black,
  title=Definition
}

\tcolorboxenvironment{example}{
  commonbox,
  colback=examplecolor!5!white,
  colframe=examplecolor!60!black,
  title=Example
}

\tcolorboxenvironment{lemma}{
  commonbox,
  colback=theoremcolor!6!white,
  colframe=theoremcolor!70!black,
  title=Lemma
}

\tcolorboxenvironment{corollary}{
  commonbox,
  colback=theoremcolor!6!white,
  colframe=theoremcolor!70!black,
  title=Corollary
}

\tcolorboxenvironment{proposition}{
  commonbox,
  colback=theoremcolor!6!white,
  colframe=theoremcolor!70!black,
  title=Proposition
}

\tcolorboxenvironment{remark}{
  commonbox,
  colback=remarkcolor!4!white,
  colframe=remarkcolor!50!black,
  title=Remark
}

% Title page information
\title{Advanced Mathematics Textbook\\
       \large A Comprehensive Guide}
\author{Your Name}
\date{\today}

% Begin document
\begin{document}

% Title page
\maketitle

% Table of contents
\frontmatter
\tableofcontents
\listoffigures
\listoftables

% Preface
\chapter*{Preface}
\addcontentsline{toc}{chapter}{Preface}

This textbook provides a comprehensive introduction to advanced mathematical concepts. Each chapter builds upon previous knowledge while introducing new ideas and techniques essential for higher mathematics.

The book is organized into several major areas of mathematics, from foundational algebra and calculus through more advanced topics in analysis, linear algebra, and discrete mathematics. The approach follows classical treatments found in works such as Rudin \cite{rudin1976principles} for analysis, Dummit and Foote \cite{dummit2004abstract} for abstract algebra, and Strang \cite{strang2016introduction} for linear algebra.

Mathematical rigor is emphasized throughout, with careful attention to definitions, theorems, and proofs as advocated by Spivak \cite{spivak2008calculus}. The probability chapter draws from the foundational work of Ross \cite{ross2014first}, while discrete mathematics concepts follow the comprehensive treatment in Rosen \cite{rosen2019discrete}.

\mainmatter

% Include chapters
\chapter{Mathematical Foundations}

This chapter establishes the fundamental concepts and notation that will be used throughout the textbook. We begin with set theory, logic, and proof techniques.

\section{Set Theory}

\begin{definition}[Set]
A \textbf{set} is a well-defined collection of distinct objects, called elements or members of the set.
\end{definition}

\begin{example}
The set of natural numbers less than 5 can be written as:
\[N_5 = \{1, 2, 3, 4\}\]
\end{example}

\subsection{Set Operations}

Let $A$ and $B$ be sets. We define the following operations:

\begin{definition}[Union]
The \textbf{union} of sets $A$ and $B$ is:
\[A \cup B = \{x : x \in A \text{ or } x \in B\}\]
\end{definition}

\begin{definition}[Intersection]
The \textbf{intersection} of sets $A$ and $B$ is:
\[A \cap B = \{x : x \in A \text{ and } x \in B\}\]
\end{definition}

\begin{definition}[Complement]
The \textbf{complement} of set $A$ with respect to universal set $U$ is:
\[A^c = \{x \in U : x \notin A\}\]
\end{definition}

\section{Logic and Proof Techniques}

\subsection{Logical Statements}

\begin{definition}[Proposition]
A \textbf{proposition} is a statement that is either true or false, but not both.
\end{definition}

\begin{example}
The following are propositions:
\begin{itemize}
    \item $2 + 2 = 4$ (True)
    \item All prime numbers are odd (False, since 2 is prime and even)
    \item $\sqrt{2}$ is irrational (True)
\end{itemize}
\end{example}

\subsection{Methods of Proof}

\begin{definition}[Direct Proof]
A \textbf{direct proof} of a statement "If $P$, then $Q$" assumes $P$ is true and uses logical steps to show that $Q$ must be true.
\end{definition}

\begin{theorem}
If $n$ is an even integer, then $n^2$ is even.
\end{theorem}

\begin{proof}
Let $n$ be an even integer. Then $n = 2k$ for some integer $k$. We have:
\[n^2 = (2k)^2 = 4k^2 = 2(2k^2)\]
Since $2k^2$ is an integer, $n^2 = 2(2k^2)$ is even.
\end{proof}

\begin{definition}[Proof by Contradiction]
A \textbf{proof by contradiction} assumes the negation of what we want to prove and shows this leads to a logical contradiction.
\end{definition}

\begin{theorem}
$\sqrt{2}$ is irrational.
\end{theorem}

\begin{proof}
Assume for contradiction that $\sqrt{2}$ is rational. Then $\sqrt{2} = \frac{p}{q}$ where $p$ and $q$ are integers with $\gcd(p,q) = 1$.

Squaring both sides: $2 = \frac{p^2}{q^2}$, so $2q^2 = p^2$.

This means $p^2$ is even, which implies $p$ is even. Let $p = 2r$ for some integer $r$.

Substituting: $2q^2 = (2r)^2 = 4r^2$, so $q^2 = 2r^2$.

This means $q^2$ is even, which implies $q$ is even.

But if both $p$ and $q$ are even, then $\gcd(p,q) \geq 2$, contradicting our assumption that $\gcd(p,q) = 1$.

Therefore, $\sqrt{2}$ is irrational.
\end{proof}

\section{Functions and Relations}

\begin{definition}[Function]
A \textbf{function} $f: A \to B$ is a relation that assigns to each element $a \in A$ exactly one element $f(a) \in B$.
\end{definition}

\begin{definition}[Injective Function]
A function $f: A \to B$ is \textbf{injective} (one-to-one) if for all $a_1, a_2 \in A$:
\[f(a_1) = f(a_2) \implies a_1 = a_2\]
\end{definition}

\begin{definition}[Surjective Function]
A function $f: A \to B$ is \textbf{surjective} (onto) if for all $b \in B$, there exists $a \in A$ such that $f(a) = b$.
\end{definition}

\begin{definition}[Bijective Function]
A function is \textbf{bijective} if it is both injective and surjective.
\end{definition}

\section{Exercises}

\begin{enumerate}
    \item Prove that the intersection of two sets is commutative: $A \cap B = B \cap A$.
    
    \item Show that if $f: A \to B$ and $g: B \to C$ are both injective, then $g \circ f: A \to C$ is injective.
    
    \item Prove by contradiction that there are infinitely many prime numbers.
    
    \item Let $A = \{1, 2, 3, 4\}$ and $B = \{2, 4, 6, 8\}$. Find $A \cup B$, $A \cap B$, and $A \setminus B$.
\end{enumerate}

\chapter{Abstract Algebra}

This chapter introduces the fundamental algebraic structures that form the foundation of modern algebra.

\section{Groups}

\begin{definition}[Binary Operation]
A \textbf{binary operation} on a set $S$ is a function $*: S \times S \to S$ that assigns to each ordered pair $(a,b)$ of elements in $S$ an element $a * b$ in $S$.
\end{definition}

\begin{definition}[Group]
A \textbf{group} is a set $G$ together with a binary operation $*$ such that:
\begin{enumerate}
    \item \textbf{Closure:} For all $a, b \in G$, we have $a * b \in G$.
    \item \textbf{Associativity:} For all $a, b, c \in G$, we have $(a * b) * c = a * (b * c)$.
    \item \textbf{Identity:} There exists an element $e \in G$ such that for all $a \in G$, $e * a = a * e = a$.
    \item \textbf{Inverse:} For each $a \in G$, there exists an element $a^{-1} \in G$ such that $a * a^{-1} = a^{-1} * a = e$.
\end{enumerate}
\end{definition}

\begin{example}[The Integers under Addition]
The set $\mathbb{Z}$ with the operation of addition forms a group:
\begin{itemize}
    \item Closure: If $a, b \in \mathbb{Z}$, then $a + b \in \mathbb{Z}$
    \item Associativity: $(a + b) + c = a + (b + c)$ for all $a, b, c \in \mathbb{Z}$
    \item Identity: $0$ is the identity element since $a + 0 = 0 + a = a$
    \item Inverse: For each $a \in \mathbb{Z}$, the inverse is $-a$ since $a + (-a) = 0$
\end{itemize}
\end{example}

\begin{theorem}
In any group $G$, the identity element is unique.
\end{theorem}

\begin{proof}
Suppose $e$ and $e'$ are both identity elements in $G$. Then:
\[e = e * e' = e'\]
where the first equality uses the fact that $e'$ is an identity, and the second equality uses the fact that $e$ is an identity.
\end{proof}

\subsection{Subgroups}

\begin{definition}[Subgroup]
Let $(G, *)$ be a group. A subset $H \subseteq G$ is a \textbf{subgroup} of $G$ if $(H, *)$ is itself a group.
\end{definition}

\begin{theorem}[Subgroup Test]
A non-empty subset $H$ of a group $G$ is a subgroup if and only if:
\begin{enumerate}
    \item For all $a, b \in H$, we have $ab \in H$ (closure)
    \item For all $a \in H$, we have $a^{-1} \in H$ (inverse)
\end{enumerate}
\end{theorem}

\section{Rings}

\begin{definition}[Ring]
A \textbf{ring} is a set $R$ with two binary operations, addition $(+)$ and multiplication $(\cdot)$, such that:
\begin{enumerate}
    \item $(R, +)$ is an abelian group
    \item Multiplication is associative
    \item The distributive laws hold: $a \cdot (b + c) = a \cdot b + a \cdot c$ and $(a + b) \cdot c = a \cdot c + b \cdot c$
\end{enumerate}
\end{definition}

\begin{example}[The Integers]
The set $\mathbb{Z}$ with ordinary addition and multiplication forms a ring.
\end{example}

\begin{definition}[Field]
A \textbf{field} is a commutative ring with unity in which every non-zero element has a multiplicative inverse.
\end{definition}

\begin{example}[The Rational Numbers]
The set $\mathbb{Q}$ of rational numbers forms a field under ordinary addition and multiplication.
\end{example}

\section{Vector Spaces}

\begin{definition}[Vector Space]
Let $F$ be a field. A \textbf{vector space} over $F$ is a set $V$ together with operations of vector addition and scalar multiplication such that:
\begin{enumerate}
    \item $(V, +)$ is an abelian group
    \item For all $\alpha \in F$ and $\mathbf{v} \in V$, we have $\alpha \mathbf{v} \in V$
    \item $\alpha(\mathbf{u} + \mathbf{v}) = \alpha\mathbf{u} + \alpha\mathbf{v}$
    \item $(\alpha + \beta)\mathbf{v} = \alpha\mathbf{v} + \beta\mathbf{v}$
    \item $(\alpha\beta)\mathbf{v} = \alpha(\beta\mathbf{v})$
    \item $1\mathbf{v} = \mathbf{v}$ where $1$ is the multiplicative identity in $F$
\end{enumerate}
\end{definition}

\begin{example}[Euclidean Space]
The set $\mathbb{R}^n$ with componentwise addition and scalar multiplication forms a vector space over $\mathbb{R}$.
\end{example}

\begin{definition}[Linear Independence]
Vectors $\mathbf{v_1}, \mathbf{v_2}, \ldots, \mathbf{v_n}$ in a vector space $V$ are \textbf{linearly independent} if the only solution to
\[\alpha_1 \mathbf{v_1} + \alpha_2 \mathbf{v_2} + \cdots + \alpha_n \mathbf{v_n} = \mathbf{0}\]
is $\alpha_1 = \alpha_2 = \cdots = \alpha_n = 0$.
\end{definition}

\begin{definition}[Basis]
A \textbf{basis} for a vector space $V$ is a set of vectors that is both linearly independent and spans $V$.
\end{definition}

\section{Exercises}

\begin{enumerate}
    \item Prove that in any group, each element has a unique inverse.
    
    \item Show that the set of even integers forms a subgroup of $(\mathbb{Z}, +)$.
    
    \item Verify that the set $\mathbb{Z}/n\mathbb{Z}$ forms a ring under addition and multiplication modulo $n$.
    
    \item Prove that if $\mathbf{v_1}, \mathbf{v_2}, \ldots, \mathbf{v_n}$ are linearly independent, then any subset is also linearly independent.
    
    \item Find a basis for the vector space of $2 \times 2$ matrices over $\mathbb{R}$.
\end{enumerate}

\chapter{Calculus}

This chapter covers the fundamental concepts of differential and integral calculus, building upon the foundations established in previous chapters.

\section{Limits and Continuity}

\begin{definition}[Limit]
Let $f$ be a function defined on some open interval containing $a$, except possibly at $a$ itself. We say that
\[\lim_{x \to a} f(x) = L\]
if for every $\varepsilon > 0$, there exists a $\delta > 0$ such that whenever $0 < |x - a| < \delta$, we have $|f(x) - L| < \varepsilon$.
\end{definition}

\begin{example}
To show that $\lim_{x \to 2} (3x - 1) = 5$:

Given $\varepsilon > 0$, we need $|(3x - 1) - 5| < \varepsilon$, which simplifies to $|3x - 6| = 3|x - 2| < \varepsilon$.

Taking $\delta = \frac{\varepsilon}{3}$, whenever $0 < |x - 2| < \delta$, we have:
\[|(3x - 1) - 5| = 3|x - 2| < 3 \cdot \frac{\varepsilon}{3} = \varepsilon\]
\end{example}

\begin{definition}[Continuity]
A function $f$ is \textbf{continuous} at $a$ if:
\[\lim_{x \to a} f(x) = f(a)\]
\end{definition}

\begin{theorem}[Intermediate Value Theorem]
If $f$ is continuous on $[a, b]$ and $k$ is any number between $f(a)$ and $f(b)$, then there exists a number $c \in (a, b)$ such that $f(c) = k$.
\end{theorem}

\section{Differentiation}

\begin{definition}[Derivative]
The \textbf{derivative} of a function $f$ at $a$ is:
\[f'(a) = \lim_{h \to 0} \frac{f(a + h) - f(a)}{h}\]
provided this limit exists.
\end{definition}

\begin{theorem}[Chain Rule]
If $g$ is differentiable at $a$ and $f$ is differentiable at $g(a)$, then the composite function $f \circ g$ is differentiable at $a$ and:
\[(f \circ g)'(a) = f'(g(a)) \cdot g'(a)\]
\end{theorem}

\begin{example}[Computing a Derivative]
Let $f(x) = \sin(x^2)$. Using the chain rule:
\[f'(x) = \cos(x^2) \cdot \frac{d}{dx}[x^2] = \cos(x^2) \cdot 2x = 2x\cos(x^2)\]
\end{example}

\subsection{Applications of Derivatives}

\begin{theorem}[Mean Value Theorem]
If $f$ is continuous on $[a, b]$ and differentiable on $(a, b)$, then there exists a number $c \in (a, b)$ such that:
\[f'(c) = \frac{f(b) - f(a)}{b - a}\]
\end{theorem}

\begin{definition}[Critical Point]
A \textbf{critical point} of a function $f$ is a number $c$ in the domain of $f$ where either $f'(c) = 0$ or $f'(c)$ does not exist.
\end{definition}

\section{Integration}

\begin{definition}[Riemann Sum]
For a function $f$ defined on $[a, b]$, a \textbf{Riemann sum} is:
\[S = \sum_{i=1}^{n} f(x_i^*) \Delta x_i\]
where $\Delta x_i = x_i - x_{i-1}$ and $x_i^* \in [x_{i-1}, x_i]$.
\end{definition}

\begin{definition}[Definite Integral]
The \textbf{definite integral} of $f$ from $a$ to $b$ is:
\[\int_a^b f(x) \, dx = \lim_{n \to \infty} \sum_{i=1}^{n} f(x_i^*) \Delta x\]
where $\Delta x = \frac{b-a}{n}$ and $x_i^* = a + i \Delta x$.
\end{definition}

\begin{theorem}[Fundamental Theorem of Calculus]
If $f$ is continuous on $[a, b]$, then:
\begin{enumerate}
    \item If $g(x) = \int_a^x f(t) \, dt$, then $g'(x) = f(x)$
    \item $\int_a^b f(x) \, dx = F(b) - F(a)$ where $F$ is any antiderivative of $f$
\end{enumerate}
\end{theorem}

\begin{example}[Definite Integral]
Evaluate $\int_0^1 x^2 \, dx$:

Since $\frac{d}{dx}\left[\frac{x^3}{3}\right] = x^2$, we have:
\[\int_0^1 x^2 \, dx = \left[\frac{x^3}{3}\right]_0^1 = \frac{1^3}{3} - \frac{0^3}{3} = \frac{1}{3}\]
\end{example}

\section{Techniques of Integration}

\subsection{Integration by Parts}

\begin{theorem}[Integration by Parts]
\[\int u \, dv = uv - \int v \, du\]
\end{theorem}

\begin{example}
Evaluate $\int x e^x \, dx$:

Let $u = x$ and $dv = e^x dx$. Then $du = dx$ and $v = e^x$.
\[\int x e^x \, dx = x e^x - \int e^x \, dx = x e^x - e^x + C = e^x(x - 1) + C\]
\end{example}

\subsection{Trigonometric Substitution}

For integrals involving $\sqrt{a^2 - x^2}$, $\sqrt{a^2 + x^2}$, or $\sqrt{x^2 - a^2}$, we can use trigonometric substitutions:

\begin{itemize}
    \item For $\sqrt{a^2 - x^2}$: use $x = a \sin \theta$
    \item For $\sqrt{a^2 + x^2}$: use $x = a \tan \theta$  
    \item For $\sqrt{x^2 - a^2}$: use $x = a \sec \theta$
\end{itemize}

\section{Series}

\begin{definition}[Infinite Series]
An \textbf{infinite series} is an expression of the form:
\[\sum_{n=1}^{\infty} a_n = a_1 + a_2 + a_3 + \cdots\]
\end{definition}

\begin{definition}[Convergence]
A series $\sum_{n=1}^{\infty} a_n$ \textbf{converges} to $S$ if:
\[\lim_{n \to \infty} S_n = S\]
where $S_n = \sum_{k=1}^{n} a_k$ is the $n$-th partial sum.
\end{definition}

\begin{theorem}[Ratio Test]
For a series $\sum a_n$ with positive terms, let $L = \lim_{n \to \infty} \frac{a_{n+1}}{a_n}$.
\begin{itemize}
    \item If $L < 1$, the series converges
    \item If $L > 1$, the series diverges
    \item If $L = 1$, the test is inconclusive
\end{itemize}
\end{theorem}

\section{Exercises}

\begin{enumerate}
    \item Use the definition of limit to prove that $\lim_{x \to 3} (2x + 1) = 7$.
    
    \item Find the derivative of $f(x) = \ln(\cos(x^2))$.
    
    \item Use the Mean Value Theorem to show that $|\sin a - \sin b| \leq |a - b|$ for all real numbers $a$ and $b$.
    
    \item Evaluate $\int_0^{\pi/2} x \sin x \, dx$ using integration by parts.
    
    \item Determine the convergence of the series $\sum_{n=1}^{\infty} \frac{n!}{n^n}$.
\end{enumerate}

\chapter{Linear Algebra}

Linear algebra studies vector spaces and linear transformations between them. This chapter covers the essential concepts and computational techniques.

\section{Matrices and Systems of Linear Equations}

\begin{definition}[Matrix]
An $m \times n$ \textbf{matrix} is a rectangular array of numbers arranged in $m$ rows and $n$ columns:
\[A = \begin{pmatrix}
a_{11} & a_{12} & \cdots & a_{1n} \\
a_{21} & a_{22} & \cdots & a_{2n} \\
\vdots & \vdots & \ddots & \vdots \\
a_{m1} & a_{m2} & \cdots & a_{mn}
\end{pmatrix}\]
\end{definition}

\begin{definition}[Matrix Addition]
If $A$ and $B$ are both $m \times n$ matrices, then $A + B$ is the $m \times n$ matrix whose $(i,j)$-entry is $a_{ij} + b_{ij}$.
\end{definition}

\begin{definition}[Matrix Multiplication]
If $A$ is an $m \times p$ matrix and $B$ is a $p \times n$ matrix, then $AB$ is the $m \times n$ matrix whose $(i,j)$-entry is:
\[(AB)_{ij} = \sum_{k=1}^{p} a_{ik}b_{kj}\]
\end{definition}

\begin{example}[Matrix Multiplication]
Let $A = \begin{pmatrix} 1 & 2 \\ 3 & 4 \end{pmatrix}$ and $B = \begin{pmatrix} 5 & 6 \\ 7 & 8 \end{pmatrix}$. Then:
\[AB = \begin{pmatrix} 1 \cdot 5 + 2 \cdot 7 & 1 \cdot 6 + 2 \cdot 8 \\ 3 \cdot 5 + 4 \cdot 7 & 3 \cdot 6 + 4 \cdot 8 \end{pmatrix} = \begin{pmatrix} 19 & 22 \\ 43 & 50 \end{pmatrix}\]
\end{example}

\subsection{Gaussian Elimination}

\begin{definition}[Row Echelon Form]
A matrix is in \textbf{row echelon form} if:
\begin{enumerate}
    \item All nonzero rows are above any rows of all zeros
    \item Each leading entry is in a column to the right of the leading entry in the row above it
    \item All entries in a column below a leading entry are zeros
\end{enumerate}
\end{definition}

\begin{definition}[Reduced Row Echelon Form]
A matrix is in \textbf{reduced row echelon form} if it is in row echelon form and:
\begin{enumerate}
    \item The leading entry in each nonzero row is 1
    \item Each leading 1 is the only nonzero entry in its column
\end{enumerate}
\end{definition}

\section{Determinants}

\begin{definition}[Determinant (2×2)]
For a $2 \times 2$ matrix $A = \begin{pmatrix} a & b \\ c & d \end{pmatrix}$:
\[\det(A) = ad - bc\]
\end{definition}

\begin{definition}[Determinant (n×n)]
For an $n \times n$ matrix $A$, the determinant can be computed using cofactor expansion:
\[\det(A) = \sum_{j=1}^{n} (-1)^{i+j} a_{ij} M_{ij}\]
where $M_{ij}$ is the $(i,j)$-minor of $A$.
\end{definition}

\begin{theorem}[Properties of Determinants]
Let $A$ and $B$ be $n \times n$ matrices. Then:
\begin{enumerate}
    \item $\det(AB) = \det(A)\det(B)$
    \item $\det(A^T) = \det(A)$
    \item If $A$ is invertible, then $\det(A^{-1}) = \frac{1}{\det(A)}$
    \item $\det(cA) = c^n \det(A)$ for scalar $c$
\end{enumerate}
\end{theorem}

\section{Vector Spaces and Subspaces}

\begin{definition}[Subspace]
A subset $W$ of a vector space $V$ is a \textbf{subspace} if:
\begin{enumerate}
    \item The zero vector is in $W$
    \item $W$ is closed under addition
    \item $W$ is closed under scalar multiplication
\end{enumerate}
\end{definition}

\begin{example}[Column Space]
The \textbf{column space} of an $m \times n$ matrix $A$ is the subspace of $\mathbb{R}^m$ spanned by the columns of $A$:
\[\text{Col}(A) = \{\mathbf{b} \in \mathbb{R}^m : A\mathbf{x} = \mathbf{b} \text{ has a solution}\}\]
\end{example}

\begin{definition}[Null Space]
The \textbf{null space} of an $m \times n$ matrix $A$ is:
\[\text{Null}(A) = \{\mathbf{x} \in \mathbb{R}^n : A\mathbf{x} = \mathbf{0}\}\]
\end{definition}

\section{Eigenvalues and Eigenvectors}

\begin{definition}[Eigenvalue and Eigenvector]
Let $A$ be an $n \times n$ matrix. A scalar $\lambda$ is an \textbf{eigenvalue} of $A$ if there exists a nonzero vector $\mathbf{v}$ such that:
\[A\mathbf{v} = \lambda\mathbf{v}\]
The vector $\mathbf{v}$ is called an \textbf{eigenvector} corresponding to $\lambda$.
\end{definition}

\begin{definition}[Characteristic Polynomial]
The \textbf{characteristic polynomial} of an $n \times n$ matrix $A$ is:
\[p(\lambda) = \det(A - \lambda I)\]
The eigenvalues of $A$ are the roots of this polynomial.
\end{definition}

\begin{example}[Finding Eigenvalues]
For $A = \begin{pmatrix} 3 & 1 \\ 0 & 2 \end{pmatrix}$:

\[A - \lambda I = \begin{pmatrix} 3-\lambda & 1 \\ 0 & 2-\lambda \end{pmatrix}\]

\[\det(A - \lambda I) = (3-\lambda)(2-\lambda) = \lambda^2 - 5\lambda + 6 = (\lambda-2)(\lambda-3)\]

So the eigenvalues are $\lambda_1 = 2$ and $\lambda_2 = 3$.
\end{example}

\begin{theorem}[Diagonalization]
An $n \times n$ matrix $A$ is diagonalizable if and only if $A$ has $n$ linearly independent eigenvectors. In this case, $A = PDP^{-1}$ where $D$ is diagonal and the columns of $P$ are eigenvectors of $A$.
\end{theorem}

\section{Inner Products and Orthogonality}

\begin{definition}[Inner Product]
An \textbf{inner product} on a real vector space $V$ is a function $\langle \cdot, \cdot \rangle : V \times V \to \mathbb{R}$ that satisfies:
\begin{enumerate}
    \item $\langle \mathbf{u}, \mathbf{v} \rangle = \langle \mathbf{v}, \mathbf{u} \rangle$
    \item $\langle \mathbf{u} + \mathbf{v}, \mathbf{w} \rangle = \langle \mathbf{u}, \mathbf{w} \rangle + \langle \mathbf{v}, \mathbf{w} \rangle$
    \item $\langle c\mathbf{u}, \mathbf{v} \rangle = c\langle \mathbf{u}, \mathbf{v} \rangle$
    \item $\langle \mathbf{v}, \mathbf{v} \rangle \geq 0$ with equality if and only if $\mathbf{v} = \mathbf{0}$
\end{enumerate}
\end{definition}

\begin{definition}[Orthogonal Vectors]
Two vectors $\mathbf{u}$ and $\mathbf{v}$ are \textbf{orthogonal} if $\langle \mathbf{u}, \mathbf{v} \rangle = 0$.
\end{definition}

\begin{definition}[Orthogonal Matrix]
A square matrix $Q$ is \textbf{orthogonal} if $Q^TQ = I$, or equivalently, $Q^{-1} = Q^T$.
\end{definition}

\begin{theorem}[Gram-Schmidt Process]
Given linearly independent vectors $\mathbf{v_1}, \mathbf{v_2}, \ldots, \mathbf{v_k}$, the Gram-Schmidt process produces orthogonal vectors $\mathbf{u_1}, \mathbf{u_2}, \ldots, \mathbf{u_k}$ such that $\text{span}\{\mathbf{u_1}, \ldots, \mathbf{u_j}\} = \text{span}\{\mathbf{v_1}, \ldots, \mathbf{v_j}\}$ for each $j$.
\end{theorem}

\section{Exercises}

\begin{enumerate}
    \item Compute $AB$ and $BA$ for $A = \begin{pmatrix} 1 & 2 & 3 \\ 4 & 5 & 6 \end{pmatrix}$ and $B = \begin{pmatrix} 1 & 4 \\ 2 & 5 \\ 3 & 6 \end{pmatrix}$.
    
    \item Find the determinant of $A = \begin{pmatrix} 2 & -1 & 3 \\ 1 & 0 & 4 \\ -2 & 1 & 1 \end{pmatrix}$.
    
    \item Determine if the vectors $\begin{pmatrix} 1 \\ 2 \\ 3 \end{pmatrix}$, $\begin{pmatrix} 4 \\ 5 \\ 6 \end{pmatrix}$, and $\begin{pmatrix} 7 \\ 8 \\ 9 \end{pmatrix}$ are linearly independent.
    
    \item Find the eigenvalues and eigenvectors of $A = \begin{pmatrix} 4 & -2 \\ 1 & 1 \end{pmatrix}$.
    
    \item Use the Gram-Schmidt process to orthogonalize the vectors $\mathbf{v_1} = \begin{pmatrix} 1 \\ 1 \\ 0 \end{pmatrix}$ and $\mathbf{v_2} = \begin{pmatrix} 1 \\ 0 \\ 1 \end{pmatrix}$.
\end{enumerate}

\chapter{Real Analysis}

Real analysis provides the rigorous foundation for calculus and studies the properties of real numbers, sequences, series, and functions.

\section{The Real Number System}

\begin{definition}[Supremum]
Let $S \subseteq \mathbb{R}$ be bounded above. The \textbf{supremum} (or least upper bound) of $S$, denoted $\sup S$, is the smallest number that is greater than or equal to every element of $S$.
\end{definition}

\begin{theorem}[Completeness Axiom]
Every non-empty subset of $\mathbb{R}$ that is bounded above has a supremum in $\mathbb{R}$.
\end{theorem}

\begin{definition}[Archimedean Property]
For any real numbers $x$ and $y$ with $x > 0$, there exists a positive integer $n$ such that $nx > y$.
\end{definition}

\section{Sequences and Series}

\begin{definition}[Convergent Sequence]
A sequence $(a_n)$ \textbf{converges} to $L$ if for every $\varepsilon > 0$, there exists $N \in \mathbb{N}$ such that for all $n \geq N$:
\[|a_n - L| < \varepsilon\]
We write $\lim_{n \to \infty} a_n = L$.
\end{definition}

\begin{theorem}[Monotone Convergence Theorem]
Every bounded monotone sequence converges.
\end{theorem}

\begin{proof}
Let $(a_n)$ be a bounded increasing sequence. Since $(a_n)$ is bounded above, by the completeness axiom, $S = \{a_n : n \in \mathbb{N}\}$ has a supremum $L = \sup S$.

For any $\varepsilon > 0$, since $L$ is the least upper bound, $L - \varepsilon$ is not an upper bound. Therefore, there exists $N$ such that $a_N > L - \varepsilon$.

Since $(a_n)$ is increasing, for all $n \geq N$:
\[L - \varepsilon < a_N \leq a_n \leq L < L + \varepsilon\]

Thus $|a_n - L| < \varepsilon$, proving convergence to $L$.
\end{proof}

\begin{definition}[Cauchy Sequence]
A sequence $(a_n)$ is \textbf{Cauchy} if for every $\varepsilon > 0$, there exists $N \in \mathbb{N}$ such that for all $m, n \geq N$:
\[|a_m - a_n| < \varepsilon\]
\end{definition}

\begin{theorem}[Cauchy Criterion]
A sequence in $\mathbb{R}$ converges if and only if it is Cauchy.
\end{theorem}

\section{Continuity and Uniform Continuity}

\begin{definition}[Continuity at a Point]
A function $f: D \to \mathbb{R}$ is \textbf{continuous at} $c \in D$ if for every $\varepsilon > 0$, there exists $\delta > 0$ such that for all $x \in D$:
\[|x - c| < \delta \implies |f(x) - f(c)| < \varepsilon\]
\end{definition}

\begin{definition}[Uniform Continuity]
A function $f: D \to \mathbb{R}$ is \textbf{uniformly continuous} on $D$ if for every $\varepsilon > 0$, there exists $\delta > 0$ such that for all $x, y \in D$:
\[|x - y| < \delta \implies |f(x) - f(y)| < \varepsilon\]
\end{definition}

\begin{theorem}[Uniform Continuity on Compact Sets]
If $f$ is continuous on a compact set $K$, then $f$ is uniformly continuous on $K$.
\end{theorem}

\section{Differentiation}

\begin{theorem}[Rolle's Theorem]
If $f$ is continuous on $[a,b]$, differentiable on $(a,b)$, and $f(a) = f(b)$, then there exists $c \in (a,b)$ such that $f'(c) = 0$.
\end{theorem}

\begin{theorem}[Mean Value Theorem]
If $f$ is continuous on $[a,b]$ and differentiable on $(a,b)$, then there exists $c \in (a,b)$ such that:
\[f'(c) = \frac{f(b) - f(a)}{b - a}\]
\end{theorem}

\begin{definition}[Uniform Convergence]
A sequence of functions $(f_n)$ \textbf{converges uniformly} to $f$ on a set $E$ if for every $\varepsilon > 0$, there exists $N \in \mathbb{N}$ such that for all $n \geq N$ and all $x \in E$:
\[|f_n(x) - f(x)| < \varepsilon\]
\end{definition}

\begin{theorem}[Uniform Convergence and Continuity]
If $(f_n)$ is a sequence of continuous functions that converges uniformly to $f$ on a set $E$, then $f$ is continuous on $E$.
\end{theorem}

\section{Integration}

\begin{definition}[Riemann Integrable]
A bounded function $f$ on $[a,b]$ is \textbf{Riemann integrable} if:
\[\overline{\int_a^b} f = \underline{\int_a^b} f\]
where $\overline{\int_a^b} f$ and $\underline{\int_a^b} f$ are the upper and lower Darboux integrals, respectively.
\end{definition}

\begin{theorem}[Riemann Integrability Criterion]
A bounded function $f$ on $[a,b]$ is Riemann integrable if and only if the set of discontinuities of $f$ has measure zero.
\end{theorem}

\begin{theorem}[Fundamental Theorem of Calculus (Rigorous)]
Let $f$ be Riemann integrable on $[a,b]$.
\begin{enumerate}
    \item If $F(x) = \int_a^x f(t) \, dt$, then $F$ is continuous on $[a,b]$.
    \item If $f$ is continuous at $c \in [a,b]$, then $F'(c) = f(c)$.
    \item If $g$ is differentiable on $[a,b]$ and $g' = f$, then $\int_a^b f(x) \, dx = g(b) - g(a)$.
\end{enumerate}
\end{theorem}

\section{Metric Spaces}

\begin{definition}[Metric Space]
A \textbf{metric space} is a set $X$ together with a function $d: X \times X \to [0, \infty)$ such that for all $x, y, z \in X$:
\begin{enumerate}
    \item $d(x, y) = 0$ if and only if $x = y$
    \item $d(x, y) = d(y, x)$ (symmetry)
    \item $d(x, z) \leq d(x, y) + d(y, z)$ (triangle inequality)
\end{enumerate}
\end{definition}

\begin{definition}[Open Ball]
In a metric space $(X, d)$, the \textbf{open ball} of radius $r > 0$ centered at $x \in X$ is:
\[B_r(x) = \{y \in X : d(x, y) < r\}\]
\end{definition}

\begin{definition}[Compact Set]
A subset $K$ of a metric space is \textbf{compact} if every open cover of $K$ has a finite subcover.
\end{definition}

\begin{theorem}[Heine-Borel Theorem]
A subset of $\mathbb{R}^n$ is compact if and only if it is closed and bounded.
\end{theorem}

\section{Exercises}

\begin{enumerate}
    \item Prove that $\lim_{n \to \infty} \frac{1}{n} = 0$ using the definition of convergence.
    
    \item Show that the sequence $a_n = \frac{n}{n+1}$ is Cauchy.
    
    \item Prove that $f(x) = x^2$ is uniformly continuous on any bounded interval $[a,b]$.
    
    \item Use the Mean Value Theorem to prove that $|\sin x - \sin y| \leq |x - y|$ for all real $x, y$.
    
    \item Show that the function $f(x) = \frac{1}{x}$ is not uniformly continuous on $(0,1)$.
\end{enumerate}

\chapter{Discrete Mathematics}

Discrete mathematics studies mathematical structures that are fundamentally discrete rather than continuous, forming the foundation for computer science and combinatorics.

\section{Graph Theory}

\begin{definition}[Graph]
A \textbf{graph} $G = (V, E)$ consists of a finite set $V$ of vertices (or nodes) and a set $E$ of edges, where each edge connects two vertices.
\end{definition}

\begin{definition}[Degree]
The \textbf{degree} of a vertex $v$, denoted $\deg(v)$, is the number of edges incident to $v$.
\end{definition}

\begin{theorem}[Handshaking Lemma]
For any graph $G = (V, E)$:
\[\sum_{v \in V} \deg(v) = 2|E|\]
\end{theorem}

\begin{proof}
Each edge contributes 1 to the degree of each of its two endpoints, so the sum of all degrees counts each edge exactly twice.
\end{proof}

\begin{definition}[Path]
A \textbf{path} in a graph is a sequence of vertices $v_1, v_2, \ldots, v_k$ such that for each $i$, there is an edge between $v_i$ and $v_{i+1}$.
\end{definition}

\begin{definition}[Connected Graph]
A graph is \textbf{connected} if there is a path between every pair of vertices.
\end{definition}

\begin{definition}[Tree]
A \textbf{tree} is a connected graph with no cycles.
\end{definition}

\begin{theorem}[Tree Characterization]
For a graph $G$ with $n$ vertices, the following are equivalent:
\begin{enumerate}
    \item $G$ is a tree
    \item $G$ is connected and has $n-1$ edges
    \item $G$ is acyclic and has $n-1$ edges
    \item There is exactly one path between any two vertices in $G$
\end{enumerate}
\end{theorem}

\section{Combinatorics}

\begin{definition}[Permutation]
A \textbf{permutation} of $n$ distinct objects is an arrangement of these objects in a specific order. The number of permutations of $n$ objects is $n!$.
\end{definition}

\begin{definition}[Combination]
A \textbf{combination} is a selection of objects where order does not matter. The number of ways to choose $k$ objects from $n$ objects is:
\[\binom{n}{k} = \frac{n!}{k!(n-k)!}\]
\end{definition}

\begin{theorem}[Binomial Theorem]
For any real numbers $x$ and $y$ and positive integer $n$:
\[(x + y)^n = \sum_{k=0}^{n} \binom{n}{k} x^{n-k} y^k\]
\end{theorem}

\begin{example}[Counting Problem]
How many ways can we arrange the letters in "MATHEMATICS"?

The word has 11 letters with repetitions: M(2), A(2), T(2), H(1), E(1), I(1), C(1), S(1).

The number of arrangements is:
\[\frac{11!}{2! \cdot 2! \cdot 2! \cdot 1! \cdot 1! \cdot 1! \cdot 1! \cdot 1!} = \frac{11!}{2^3} = \frac{39,916,800}{8} = 4,989,600\]
\end{example}

\subsection{Inclusion-Exclusion Principle}

\begin{theorem}[Inclusion-Exclusion Principle]
For finite sets $A_1, A_2, \ldots, A_n$:
\begin{align}
|A_1 \cup A_2 \cup \cdots \cup A_n| &= \sum_{i} |A_i| - \sum_{i < j} |A_i \cap A_j| \\
&\quad + \sum_{i < j < k} |A_i \cap A_j \cap A_k| - \cdots + (-1)^{n+1}|A_1 \cap A_2 \cap \cdots \cap A_n|
\end{align}
\end{theorem}

\section{Number Theory}

\begin{definition}[Divisibility]
An integer $a$ \textbf{divides} an integer $b$ (written $a \mid b$) if there exists an integer $k$ such that $b = ak$.
\end{definition}

\begin{theorem}[Division Algorithm]
For any integers $a$ and $b$ with $b > 0$, there exist unique integers $q$ and $r$ such that:
\[a = bq + r \quad \text{where } 0 \leq r < b\]
\end{theorem}

\begin{definition}[Greatest Common Divisor]
The \textbf{greatest common divisor} of integers $a$ and $b$, denoted $\gcd(a, b)$, is the largest positive integer that divides both $a$ and $b$.
\end{definition}

\begin{theorem}[Euclidean Algorithm]
The Euclidean algorithm computes $\gcd(a, b)$ by repeatedly applying:
\[\gcd(a, b) = \gcd(b, a \bmod b)\]
until one of the numbers becomes 0.
\end{theorem}

\begin{theorem}[Bézout's Identity]
For any integers $a$ and $b$, there exist integers $x$ and $y$ such that:
\[\gcd(a, b) = ax + by\]
\end{theorem}

\subsection{Prime Numbers}

\begin{definition}[Prime Number]
A positive integer $p > 1$ is \textbf{prime} if its only positive divisors are 1 and $p$.
\end{definition}

\begin{theorem}[Fundamental Theorem of Arithmetic]
Every integer greater than 1 can be expressed uniquely (up to order) as a product of prime numbers.
\end{theorem}

\begin{theorem}[Infinitude of Primes]
There are infinitely many prime numbers.
\end{theorem}

\begin{proof}[Euclid's Proof]
Suppose there are only finitely many primes $p_1, p_2, \ldots, p_k$. Consider:
\[N = p_1 p_2 \cdots p_k + 1\]

Since $N > 1$, it has a prime divisor $p$. But $p$ cannot be any of $p_1, \ldots, p_k$ since $N \equiv 1 \pmod{p_i}$ for each $i$. This contradicts our assumption that we listed all primes.
\end{proof}

\section{Recurrence Relations}

\begin{definition}[Recurrence Relation]
A \textbf{recurrence relation} is an equation that defines a sequence recursively, expressing each term in terms of previous terms.
\end{definition}

\begin{example}[Fibonacci Sequence]
The Fibonacci sequence is defined by:
\[F_0 = 0, \quad F_1 = 1, \quad F_n = F_{n-1} + F_{n-2} \text{ for } n \geq 2\]
\end{example}

\begin{theorem}[Linear Homogeneous Recurrence Relations]
The recurrence relation $a_n = c_1 a_{n-1} + c_2 a_{n-2}$ with characteristic equation $r^2 - c_1 r - c_2 = 0$ has:
\begin{itemize}
    \item If roots $r_1 \neq r_2$: general solution $a_n = A r_1^n + B r_2^n$
    \item If repeated root $r$: general solution $a_n = (A + Bn) r^n$
\end{itemize}
\end{theorem}

\section{Boolean Algebra}

\begin{definition}[Boolean Algebra]
A \textbf{Boolean algebra} is a set $B$ with two binary operations $\land$ (AND) and $\lor$ (OR), a unary operation $\neg$ (NOT), and constants 0 and 1, satisfying certain axioms.
\end{definition}

\begin{theorem}[De Morgan's Laws]
For any Boolean expressions $A$ and $B$:
\begin{align}
\neg(A \land B) &= \neg A \lor \neg B \\
\neg(A \lor B) &= \neg A \land \neg B
\end{align}
\end{theorem}

\begin{definition}[Boolean Function]
A \textbf{Boolean function} is a function $f: \{0,1\}^n \to \{0,1\}$ that maps $n$-tuples of Boolean values to a Boolean value.
\end{definition}

\section{Exercises}

\begin{enumerate}
    \item Prove that in any graph, the number of vertices with odd degree is even.
    
    \item Find the number of ways to distribute 10 identical balls into 4 distinct boxes.
    
    \item Use the inclusion-exclusion principle to find the number of integers from 1 to 100 that are divisible by 2, 3, or 5.
    
    \item Find $\gcd(252, 198)$ using the Euclidean algorithm and express it in the form $252x + 198y$.
    
    \item Solve the recurrence relation $a_n = 5a_{n-1} - 6a_{n-2}$ with initial conditions $a_0 = 1, a_1 = 0$.
\end{enumerate}
\chapter{Probability Theory}

Probability theory provides the mathematical foundation for analyzing random phenomena and uncertainty.

\section{Sample Spaces and Events}

\begin{definition}[Sample Space]
A \textbf{sample space} $\Omega$ is the set of all possible outcomes of a random experiment.
\end{definition}

\begin{definition}[Event]
An \textbf{event} is a subset of the sample space $\Omega$.
\end{definition}

\begin{example}[Coin Flipping]
For the experiment of flipping a coin twice:
\begin{itemize}
    \item Sample space: $\Omega = \{HH, HT, TH, TT\}$
    \item Event "at least one head": $A = \{HH, HT, TH\}$
    \item Event "exactly one tail": $B = \{HT, TH\}$
\end{itemize}
\end{example}

\section{Probability Measures}

\begin{definition}[Probability Measure]
A \textbf{probability measure} $P$ on a sample space $\Omega$ is a function that assigns to each event $A$ a number $P(A)$ satisfying:
\begin{enumerate}
    \item $P(A) \geq 0$ for all events $A$
    \item $P(\Omega) = 1$
    \item If $A_1, A_2, \ldots$ are pairwise disjoint events, then:
    \[P\left(\bigcup_{i=1}^{\infty} A_i\right) = \sum_{i=1}^{\infty} P(A_i)\]
\end{enumerate}
\end{definition}

\begin{theorem}[Basic Properties of Probability]
For any events $A$ and $B$:
\begin{enumerate}
    \item $P(\emptyset) = 0$
    \item $P(A^c) = 1 - P(A)$
    \item If $A \subseteq B$, then $P(A) \leq P(B)$
    \item $P(A \cup B) = P(A) + P(B) - P(A \cap B)$
\end{enumerate}
\end{theorem}

\section{Conditional Probability}

\begin{definition}[Conditional Probability]
The \textbf{conditional probability} of event $A$ given event $B$ with $P(B) > 0$ is:
\[P(A|B) = \frac{P(A \cap B)}{P(B)}\]
\end{definition}

\begin{theorem}[Law of Total Probability]
If $B_1, B_2, \ldots, B_n$ form a partition of $\Omega$ with $P(B_i) > 0$ for all $i$, then:
\[P(A) = \sum_{i=1}^{n} P(A|B_i)P(B_i)\]
\end{theorem}

\begin{theorem}[Bayes' Theorem]
If $B_1, B_2, \ldots, B_n$ form a partition of $\Omega$ with $P(B_i) > 0$ for all $i$, then:
\[P(B_j|A) = \frac{P(A|B_j)P(B_j)}{\sum_{i=1}^{n} P(A|B_i)P(B_i)}\]
\end{theorem}

\begin{example}[Medical Testing]
A disease affects 1\% of the population. A test for the disease is 95\% accurate (both sensitivity and specificity). If someone tests positive, what's the probability they have the disease?

Let $D$ = "has disease" and $T$ = "tests positive".
\begin{align}
P(D) &= 0.01, \quad P(D^c) = 0.99 \\
P(T|D) &= 0.95, \quad P(T|D^c) = 0.05
\end{align}

By Bayes' theorem:
\[P(D|T) = \frac{P(T|D)P(D)}{P(T|D)P(D) + P(T|D^c)P(D^c)} = \frac{0.95 \times 0.01}{0.95 \times 0.01 + 0.05 \times 0.99} \approx 0.161\]
\end{example}

\section{Random Variables}

\begin{definition}[Random Variable]
A \textbf{random variable} is a function $X: \Omega \to \mathbb{R}$ that assigns a real number to each outcome in the sample space.
\end{definition}

\begin{definition}[Probability Mass Function]
For a discrete random variable $X$, the \textbf{probability mass function} (PMF) is:
\[p_X(x) = P(X = x)\]
\end{definition}

\begin{definition}[Cumulative Distribution Function]
The \textbf{cumulative distribution function} (CDF) of a random variable $X$ is:
\[F_X(x) = P(X \leq x)\]
\end{definition}

\section{Expected Value and Variance}

\begin{definition}[Expected Value]
The \textbf{expected value} of a discrete random variable $X$ is:
\[E[X] = \sum_{x} x \cdot P(X = x)\]
\end{definition}

\begin{definition}[Variance]
The \textbf{variance} of a random variable $X$ is:
\[Var(X) = E[(X - E[X])^2] = E[X^2] - (E[X])^2\]
\end{definition}

\begin{theorem}[Linearity of Expectation]
For random variables $X$ and $Y$ and constants $a$ and $b$:
\[E[aX + bY] = aE[X] + bE[Y]\]
\end{theorem}

\section{Common Discrete Distributions}

\subsection{Binomial Distribution}

\begin{definition}[Binomial Distribution]
A random variable $X$ follows a \textbf{binomial distribution} with parameters $n$ and $p$, denoted $X \sim \text{Binomial}(n, p)$, if:
\[P(X = k) = \binom{n}{k} p^k (1-p)^{n-k} \quad \text{for } k = 0, 1, \ldots, n\]
\end{definition}

\begin{theorem}
If $X \sim \text{Binomial}(n, p)$, then:
\begin{align}
E[X] &= np \\
Var(X) &= np(1-p)
\end{align}
\end{theorem}

\subsection{Poisson Distribution}

\begin{definition}[Poisson Distribution]
A random variable $X$ follows a \textbf{Poisson distribution} with parameter $\lambda > 0$, denoted $X \sim \text{Poisson}(\lambda)$, if:
\[P(X = k) = \frac{\lambda^k e^{-\lambda}}{k!} \quad \text{for } k = 0, 1, 2, \ldots\]
\end{definition}

\begin{theorem}
If $X \sim \text{Poisson}(\lambda)$, then:
\begin{align}
E[X] &= \lambda \\
Var(X) &= \lambda
\end{align}
\end{theorem}

\section{Continuous Distributions}

\subsection{Normal Distribution}

\begin{definition}[Normal Distribution]
A random variable $X$ follows a \textbf{normal distribution} with parameters $\mu$ and $\sigma^2$, denoted $X \sim N(\mu, \sigma^2)$, if it has probability density function:
\[f_X(x) = \frac{1}{\sqrt{2\pi\sigma^2}} e^{-\frac{(x-\mu)^2}{2\sigma^2}}\]
\end{definition}

\begin{theorem}
If $X \sim N(\mu, \sigma^2)$, then:
\begin{align}
E[X] &= \mu \\
Var(X) &= \sigma^2
\end{align}
\end{theorem}

\begin{theorem}[Central Limit Theorem]
Let $X_1, X_2, \ldots, X_n$ be independent and identically distributed random variables with mean $\mu$ and variance $\sigma^2$. Then:
\[\frac{\bar{X}_n - \mu}{\sigma/\sqrt{n}} \xrightarrow{d} N(0, 1) \quad \text{as } n \to \infty\]
where $\bar{X}_n = \frac{1}{n}\sum_{i=1}^n X_i$.
\end{theorem}

\section{Limit Theorems}

\begin{theorem}[Law of Large Numbers]
Let $X_1, X_2, \ldots$ be independent and identically distributed random variables with finite mean $\mu$. Then:
\[\bar{X}_n = \frac{1}{n}\sum_{i=1}^n X_i \to \mu \quad \text{as } n \to \infty\]
\end{theorem}

\section{Exercises}

\begin{enumerate}
    \item A fair six-sided die is rolled twice. Find the probability that the sum is 7 given that at least one roll shows a 3.
    
    \item If $X \sim \text{Binomial}(10, 0.3)$, compute $P(X = 3)$ and $E[X]$.
    
    \item Customers arrive at a store according to a Poisson process with rate 2 per hour. What is the probability that exactly 3 customers arrive in a 2-hour period?
    
    \item If $X \sim N(50, 100)$, find $P(40 < X < 60)$.
    
    \item Use Bayes' theorem to solve: A bag contains 3 red balls and 2 blue balls. A ball is drawn and replaced 3 times, with 2 reds and 1 blue observed. What's the probability the bag actually contains 4 red balls and 1 blue ball?
\end{enumerate}

\chapter{Geometry}

This chapter explores the fundamental concepts of Euclidean geometry, coordinate geometry, and introduces elements of non-Euclidean geometry.

\section{Euclidean Geometry}

\begin{definition}[Point, Line, Plane]
The fundamental objects of Euclidean geometry are:
\begin{itemize}
    \item A \textbf{point} has no dimension (position only)
    \item A \textbf{line} is one-dimensional and extends infinitely in both directions
    \item A \textbf{plane} is two-dimensional and extends infinitely in all directions
\end{itemize}
\end{definition}

\subsection{Euclid's Axioms}

\begin{theorem}[Euclid's Five Postulates]
\begin{enumerate}
    \item A straight line can be drawn between any two points
    \item Any finite straight line can be extended indefinitely
    \item A circle can be drawn with any center and any radius
    \item All right angles are equal to each other
    \item If a line intersects two other lines such that the sum of interior angles on one side is less than two right angles, then the two lines will intersect on that side when extended
\end{enumerate}
\end{theorem}

\begin{theorem}[Parallel Postulate Equivalent]
Through a point not on a given line, there exists exactly one line parallel to the given line.
\end{theorem}

\section{Triangles and Congruence}

\begin{definition}[Triangle]
A \textbf{triangle} is a polygon with three vertices and three sides.
\end{definition}

\begin{theorem}[Triangle Congruence Criteria]
Two triangles are congruent if any of the following conditions hold:
\begin{enumerate}
    \item \textbf{SSS:} Three sides are equal
    \item \textbf{SAS:} Two sides and the included angle are equal
    \item \textbf{ASA:} Two angles and the included side are equal
    \item \textbf{AAS:} Two angles and a non-included side are equal
    \item \textbf{RHS:} Right angle, hypotenuse, and one side are equal
\end{enumerate}
\end{theorem}

\begin{theorem}[Pythagorean Theorem]
In a right triangle with legs of length $a$ and $b$ and hypotenuse of length $c$:
\[a^2 + b^2 = c^2\]
\end{theorem}

\begin{proof}
Consider a square with side length $(a+b)$ containing four copies of the right triangle. The area can be computed in two ways:
\begin{align}
(a+b)^2 &= 4 \cdot \frac{1}{2}ab + c^2 \\
a^2 + 2ab + b^2 &= 2ab + c^2 \\
a^2 + b^2 &= c^2
\end{align}
\end{proof}

\section{Circles}

\begin{definition}[Circle]
A \textbf{circle} is the set of all points in a plane that are equidistant from a fixed point called the center.
\end{definition}

\begin{theorem}[Inscribed Angle Theorem]
An inscribed angle is half the central angle that subtends the same arc.
\end{theorem}

\begin{theorem}[Power of a Point]
For a point $P$ and a circle with center $O$ and radius $r$, the power of $P$ is $|PO|^2 - r^2$. For any line through $P$ intersecting the circle at points $A$ and $B$:
\[PA \cdot PB = ||PO|^2 - r^2|\]
\end{theorem}

\section{Coordinate Geometry}

\begin{definition}[Distance Formula]
The distance between points $(x_1, y_1)$ and $(x_2, y_2)$ is:
\[d = \sqrt{(x_2 - x_1)^2 + (y_2 - y_1)^2}\]
\end{definition}

\begin{definition}[Equation of a Line]
A line can be represented by:
\begin{itemize}
    \item \textbf{Slope-intercept form:} $y = mx + b$
    \item \textbf{Point-slope form:} $y - y_1 = m(x - x_1)$
    \item \textbf{General form:} $Ax + By + C = 0$
\end{itemize}
\end{definition}

\begin{definition}[Equation of a Circle]
A circle with center $(h, k)$ and radius $r$ has equation:
\[(x - h)^2 + (y - k)^2 = r^2\]
\end{definition}

\section{Transformations}

\begin{definition}[Rigid Transformations]
\textbf{Rigid transformations} preserve distances and angles:
\begin{itemize}
    \item \textbf{Translation:} $(x, y) \mapsto (x + a, y + b)$
    \item \textbf{Rotation:} $(x, y) \mapsto (x\cos\theta - y\sin\theta, x\sin\theta + y\cos\theta)$
    \item \textbf{Reflection:} $(x, y) \mapsto (-x, y)$ (across $y$-axis)
\end{itemize}
\end{definition}

\begin{definition}[Similarity Transformations]
\textbf{Similarity transformations} preserve angles but may change distances by a constant factor:
\begin{itemize}
    \item \textbf{Scaling:} $(x, y) \mapsto (kx, ky)$ for some $k > 0$
    \item \textbf{Homothety:} Combination of scaling and translation
\end{itemize}
\end{definition}

\section{Area and Volume}

\begin{theorem}[Area Formulas]
\begin{itemize}
    \item \textbf{Triangle:} $A = \frac{1}{2}bh$ or $A = \sqrt{s(s-a)(s-b)(s-c)}$ (Heron's formula)
    \item \textbf{Rectangle:} $A = lw$
    \item \textbf{Circle:} $A = \pi r^2$
    \item \textbf{Ellipse:} $A = \pi ab$ where $a$ and $b$ are the semi-axes
\end{itemize}
\end{theorem}

\begin{theorem}[Volume Formulas]
\begin{itemize}
    \item \textbf{Rectangular prism:} $V = lwh$
    \item \textbf{Cylinder:} $V = \pi r^2 h$
    \item \textbf{Sphere:} $V = \frac{4}{3}\pi r^3$
    \item \textbf{Cone:} $V = \frac{1}{3}\pi r^2 h$
\end{itemize}
\end{theorem}

\section{Vectors in Geometry}

\begin{definition}[Vector]
A \textbf{vector} is a quantity with both magnitude and direction, often represented as $\overrightarrow{AB}$ or $\mathbf{v} = \langle a, b \rangle$.
\end{definition}

\begin{definition}[Dot Product]
For vectors $\mathbf{u} = \langle u_1, u_2 \rangle$ and $\mathbf{v} = \langle v_1, v_2 \rangle$:
\[\mathbf{u} \cdot \mathbf{v} = u_1 v_1 + u_2 v_2 = |\mathbf{u}||\mathbf{v}|\cos\theta\]
where $\theta$ is the angle between the vectors.
\end{definition}

\begin{definition}[Cross Product]
For vectors $\mathbf{u} = \langle u_1, u_2, u_3 \rangle$ and $\mathbf{v} = \langle v_1, v_2, v_3 \rangle$:
\[\mathbf{u} \times \mathbf{v} = \langle u_2 v_3 - u_3 v_2, u_3 v_1 - u_1 v_3, u_1 v_2 - u_2 v_1 \rangle\]
\end{definition}

\section{Introduction to Non-Euclidean Geometry}

\begin{definition}[Hyperbolic Geometry]
\textbf{Hyperbolic geometry} is a non-Euclidean geometry where the parallel postulate is replaced with: Through a point not on a given line, there exist infinitely many lines parallel to the given line.
\end{definition}

\begin{definition}[Spherical Geometry]
\textbf{Spherical geometry} is the geometry on the surface of a sphere, where "lines" are great circles and there are no parallel lines.
\end{definition}

\begin{theorem}[Gauss-Bonnet Theorem (Simple Form)]
For a triangle on a sphere with angles $\alpha$, $\beta$, and $\gamma$:
\[\alpha + \beta + \gamma = \pi + \frac{A}{R^2}\]
where $A$ is the area of the triangle and $R$ is the radius of the sphere.
\end{theorem}

\section{Exercises}

\begin{enumerate}
    \item Prove that the diagonals of a rhombus are perpendicular bisectors of each other.
    
    \item Find the equation of the circle passing through points $(1, 2)$, $(3, 4)$, and $(5, 2)$.
    
    \item Use vectors to prove that the diagonals of a parallelogram bisect each other.
    
    \item A triangle has sides of length 3, 4, and 5. Find its area using both the base-height formula and Heron's formula.
    
    \item In spherical geometry, what is the sum of angles in a triangle whose area is $\frac{1}{4}$ the area of a hemisphere of radius 1?
\end{enumerate}


% Appendices
\appendix
\chapter{Mathematical Notation}

This appendix provides a comprehensive reference for the mathematical notation used throughout this textbook.

\section{Set Theory Notation}

\begin{tabular}{|l|l|}
\hline
\textbf{Symbol} & \textbf{Meaning} \\
\hline
$\in$ & Element of (belongs to) \\
$\notin$ & Not an element of \\
$\subset$ & Subset of \\
$\subseteq$ & Subset of or equal to \\
$\supset$ & Superset of \\
$\supseteq$ & Superset of or equal to \\
$\cup$ & Union \\
$\cap$ & Intersection \\
$\setminus$ & Set difference \\
$A^c$ & Complement of set $A$ \\
$\emptyset$ & Empty set \\
$\mathbb{U}$ & Universal set \\
$|A|$ & Cardinality of set $A$ \\
$\mathcal{P}(A)$ & Power set of $A$ \\
\hline
\end{tabular}

\section{Number Systems}

\begin{tabular}{|l|l|}
\hline
\textbf{Symbol} & \textbf{Meaning} \\
\hline
$\mathbb{N}$ & Natural numbers $\{1, 2, 3, \ldots\}$ \\
$\mathbb{N}_0$ & Natural numbers including zero $\{0, 1, 2, 3, \ldots\}$ \\
$\mathbb{Z}$ & Integers $\{\ldots, -2, -1, 0, 1, 2, \ldots\}$ \\
$\mathbb{Q}$ & Rational numbers \\
$\mathbb{R}$ & Real numbers \\
$\mathbb{C}$ & Complex numbers \\
$\mathbb{Z}^+$ & Positive integers \\
$\mathbb{R}^+$ & Positive real numbers \\
\hline
\end{tabular}

\section{Logic and Proof Notation}

\begin{tabular}{|l|l|}
\hline
\textbf{Symbol} & \textbf{Meaning} \\
\hline
$\land$ & Logical AND \\
$\lor$ & Logical OR \\
$\neg$ & Logical NOT \\
$\implies$ & Implies \\
$\iff$ & If and only if \\
$\forall$ & For all (universal quantifier) \\
$\exists$ & There exists (existential quantifier) \\
$\exists!$ & There exists a unique \\
$\therefore$ & Therefore \\
$\because$ & Because \\
$\square$ & End of proof \\
\hline
\end{tabular}

\section{Functions and Relations}

\begin{tabular}{|l|l|}
\hline
\textbf{Symbol} & \textbf{Meaning} \\
\hline
$f: A \to B$ & Function from set $A$ to set $B$ \\
$f(x)$ & Value of function $f$ at $x$ \\
$f^{-1}$ & Inverse function of $f$ \\
$f \circ g$ & Composition of functions $f$ and $g$ \\
$\text{dom}(f)$ & Domain of function $f$ \\
$\text{ran}(f)$ & Range of function $f$ \\
$f|_A$ & Restriction of $f$ to set $A$ \\
\hline
\end{tabular}

\section{Calculus Notation}

\begin{tabular}{|l|l|}
\hline
\textbf{Symbol} & \textbf{Meaning} \\
\hline
$\lim_{x \to a} f(x)$ & Limit of $f(x)$ as $x$ approaches $a$ \\
$\lim_{x \to a^+} f(x)$ & Right-hand limit \\
$\lim_{x \to a^-} f(x)$ & Left-hand limit \\
$f'(x)$ & Derivative of $f$ with respect to $x$ \\
$\frac{df}{dx}$ & Derivative of $f$ with respect to $x$ \\
$\frac{d^n f}{dx^n}$ & $n$-th derivative of $f$ \\
$\int f(x) \, dx$ & Indefinite integral of $f$ \\
$\int_a^b f(x) \, dx$ & Definite integral from $a$ to $b$ \\
$\sum_{i=1}^n a_i$ & Sum from $i=1$ to $n$ \\
$\prod_{i=1}^n a_i$ & Product from $i=1$ to $n$ \\
\hline
\end{tabular}

\section{Linear Algebra Notation}

\begin{tabular}{|l|l|}
\hline
\textbf{Symbol} & \textbf{Meaning} \\
\hline
$\mathbf{v}$ & Vector $\mathbf{v}$ \\
$|\mathbf{v}|$ or $\|\mathbf{v}\|$ & Magnitude (norm) of vector $\mathbf{v}$ \\
$\mathbf{u} \cdot \mathbf{v}$ & Dot product of vectors $\mathbf{u}$ and $\mathbf{v}$ \\
$\mathbf{u} \times \mathbf{v}$ & Cross product of vectors $\mathbf{u}$ and $\mathbf{v}$ \\
$A^T$ & Transpose of matrix $A$ \\
$A^{-1}$ & Inverse of matrix $A$ \\
$\det(A)$ & Determinant of matrix $A$ \\
$\text{tr}(A)$ & Trace of matrix $A$ \\
$\text{rank}(A)$ & Rank of matrix $A$ \\
$\text{null}(A)$ & Null space of matrix $A$ \\
$\text{col}(A)$ & Column space of matrix $A$ \\
$I$ & Identity matrix \\
$\mathbf{0}$ & Zero vector or zero matrix \\
\hline
\end{tabular}

\section{Probability Notation}

\begin{tabular}{|l|l|}
\hline
\textbf{Symbol} & \textbf{Meaning} \\
\hline
$P(A)$ & Probability of event $A$ \\
$P(A|B)$ & Conditional probability of $A$ given $B$ \\
$A \cap B$ & Intersection of events $A$ and $B$ \\
$A \cup B$ & Union of events $A$ and $B$ \\
$A^c$ & Complement of event $A$ \\
$\Omega$ & Sample space \\
$X$ & Random variable \\
$E[X]$ & Expected value of random variable $X$ \\
$\text{Var}(X)$ & Variance of random variable $X$ \\
$\sigma_X$ & Standard deviation of random variable $X$ \\
$X \sim D$ & Random variable $X$ follows distribution $D$ \\
$F_X(x)$ & Cumulative distribution function \\
$f_X(x)$ & Probability density function \\
$p_X(x)$ & Probability mass function \\
\hline
\end{tabular}

\section{Number Theory Notation}

\begin{tabular}{|l|l|}
\hline
\textbf{Symbol} & \textbf{Meaning} \\
\hline
$a | b$ & $a$ divides $b$ \\
$a \nmid b$ & $a$ does not divide $b$ \\
$\gcd(a, b)$ & Greatest common divisor of $a$ and $b$ \\
$\text{lcm}(a, b)$ & Least common multiple of $a$ and $b$ \\
$a \equiv b \pmod{n}$ & $a$ is congruent to $b$ modulo $n$ \\
$a \bmod n$ & Remainder when $a$ is divided by $n$ \\
$\phi(n)$ & Euler's totient function \\
$\mathbb{Z}_n$ & Integers modulo $n$ \\
\hline
\end{tabular}

\section{Graph Theory Notation}

\begin{tabular}{|l|l|}
\hline
\textbf{Symbol} & \textbf{Meaning} \\
\hline
$G = (V, E)$ & Graph with vertex set $V$ and edge set $E$ \\
$|V|$ & Number of vertices \\
$|E|$ & Number of edges \\
$\deg(v)$ & Degree of vertex $v$ \\
$d(u, v)$ & Distance between vertices $u$ and $v$ \\
$K_n$ & Complete graph on $n$ vertices \\
$C_n$ & Cycle graph on $n$ vertices \\
$P_n$ & Path graph on $n$ vertices \\
\hline
\end{tabular}

\section{Common Mathematical Constants}

\begin{tabular}{|l|l|l|}
\hline
\textbf{Symbol} & \textbf{Name} & \textbf{Approximate Value} \\
\hline
$\pi$ & Pi & 3.14159... \\
$e$ & Euler's number & 2.71828... \\
$\phi$ & Golden ratio & 1.61803... \\
$\gamma$ & Euler-Mascheroni constant & 0.57721... \\
$\sqrt{2}$ & Square root of 2 & 1.41421... \\
\hline
\end{tabular}

\chapter{Selected Solutions}

This appendix provides detailed solutions to selected exercises from each chapter.

\section{Chapter 1: Mathematical Foundations}

\subsection{Exercise 1}
\textbf{Problem:} Prove that the intersection of two sets is commutative: $A \cap B = B \cap A$.

\textbf{Solution:}
We need to show that $A \cap B = B \cap A$ by proving two inclusions.

First, we show $A \cap B \subseteq B \cap A$:
Let $x \in A \cap B$. By definition of intersection, $x \in A$ and $x \in B$.
Since $x \in B$ and $x \in A$, we have $x \in B \cap A$.
Therefore, $A \cap B \subseteq B \cap A$.

Next, we show $B \cap A \subseteq A \cap B$:
Let $x \in B \cap A$. By definition of intersection, $x \in B$ and $x \in A$.
Since $x \in A$ and $x \in B$, we have $x \in A \cap B$.
Therefore, $B \cap A \subseteq A \cap B$.

Since both inclusions hold, $A \cap B = B \cap A$.

\subsection{Exercise 3}
\textbf{Problem:} Prove by contradiction that there are infinitely many prime numbers.

\textbf{Solution:}
Assume for contradiction that there are only finitely many prime numbers. Let these primes be $p_1, p_2, \ldots, p_k$.

Consider the number $N = p_1 \cdot p_2 \cdot \ldots \cdot p_k + 1$.

Since $N > 1$, by the fundamental theorem of arithmetic, $N$ must have at least one prime divisor. Let $p$ be a prime divisor of $N$.

If $p$ is one of $p_1, p_2, \ldots, p_k$, then $p$ divides the product $p_1 \cdot p_2 \cdot \ldots \cdot p_k$. Since $p$ also divides $N$, it must divide their difference:
$N - p_1 \cdot p_2 \cdot \ldots \cdot p_k = 1$

But no prime can divide 1, which is a contradiction.

Therefore, $p$ cannot be any of $p_1, p_2, \ldots, p_k$, meaning we have found a new prime not in our original list. This contradicts our assumption that we had listed all primes.

Therefore, there must be infinitely many prime numbers.

\section{Chapter 2: Abstract Algebra}

\subsection{Exercise 1}
\textbf{Problem:} Prove that in any group, each element has a unique inverse.

\textbf{Solution:}
Let $G$ be a group with operation $*$ and identity element $e$. Let $a \in G$.

Suppose $b$ and $c$ are both inverses of $a$. Then:
- $a * b = b * a = e$
- $a * c = c * a = e$

We need to show $b = c$.

Starting with $b$:
\begin{align}
b &= b * e \quad \text{(identity property)} \\
&= b * (a * c) \quad \text{(since $a * c = e$)} \\
&= (b * a) * c \quad \text{(associativity)} \\
&= e * c \quad \text{(since $b * a = e$)} \\
&= c \quad \text{(identity property)}
\end{align}

Therefore, $b = c$, proving uniqueness of the inverse.

\section{Chapter 3: Calculus}

\subsection{Exercise 1}
\textbf{Problem:} Use the definition of limit to prove that $\lim_{x \to 3} (2x + 1) = 7$.

\textbf{Solution:}
We need to show that for every $\varepsilon > 0$, there exists $\delta > 0$ such that whenever $0 < |x - 3| < \delta$, we have $|(2x + 1) - 7| < \varepsilon$.

First, let's simplify the expression we need to bound:
$|(2x + 1) - 7| = |2x - 6| = 2|x - 3|$

For this to be less than $\varepsilon$, we need:
$2|x - 3| < \varepsilon$
$|x - 3| < \frac{\varepsilon}{2}$

Therefore, we can choose $\delta = \frac{\varepsilon}{2}$.

Verification: If $0 < |x - 3| < \delta = \frac{\varepsilon}{2}$, then:
$|(2x + 1) - 7| = 2|x - 3| < 2 \cdot \frac{\varepsilon}{2} = \varepsilon$

This proves that $\lim_{x \to 3} (2x + 1) = 7$.

\subsection{Exercise 4}
\textbf{Problem:} Evaluate $\int_0^{\pi/2} x \sin x \, dx$ using integration by parts.

\textbf{Solution:}
Using integration by parts with $u = x$ and $dv = \sin x \, dx$:
- $du = dx$
- $v = -\cos x$

Applying the integration by parts formula:
\begin{align}
\int_0^{\pi/2} x \sin x \, dx &= \left[x(-\cos x)\right]_0^{\pi/2} - \int_0^{\pi/2} (-\cos x) \, dx \\
&= \left[-x \cos x\right]_0^{\pi/2} + \int_0^{\pi/2} \cos x \, dx \\
&= \left[-x \cos x\right]_0^{\pi/2} + \left[\sin x\right]_0^{\pi/2} \\
&= \left[-\frac{\pi}{2} \cos\frac{\pi}{2} - (-0 \cos 0)\right] + \left[\sin\frac{\pi}{2} - \sin 0\right] \\
&= [0 - 0] + [1 - 0] \\
&= 1
\end{align}

Therefore, $\int_0^{\pi/2} x \sin x \, dx = 1$.

\section{Chapter 4: Linear Algebra}

\subsection{Exercise 2}
\textbf{Problem:} Find the determinant of $A = \begin{pmatrix} 2 & -1 & 3 \\ 1 & 0 & 4 \\ -2 & 1 & 1 \end{pmatrix}$.

\textbf{Solution:}
Using cofactor expansion along the second row (which has a zero):

$\det(A) = 1 \cdot \det\begin{pmatrix} -1 & 3 \\ 1 & 1 \end{pmatrix} - 0 \cdot (\text{something}) + 4 \cdot \det\begin{pmatrix} 2 & -1 \\ -2 & 1 \end{pmatrix}$

Computing the 2×2 determinants:
- $\det\begin{pmatrix} -1 & 3 \\ 1 & 1 \end{pmatrix} = (-1)(1) - (3)(1) = -1 - 3 = -4$
- $\det\begin{pmatrix} 2 & -1 \\ -2 & 1 \end{pmatrix} = (2)(1) - (-1)(-2) = 2 - 2 = 0$

Therefore:
$\det(A) = 1 \cdot (-4) - 0 + 4 \cdot 0 = -4$

\section{Chapter 5: Real Analysis}

\subsection{Exercise 1}
\textbf{Problem:} Prove that $\lim_{n \to \infty} \frac{1}{n} = 0$ using the definition of convergence.

\textbf{Solution:}
We need to show that for every $\varepsilon > 0$, there exists $N \in \mathbb{N}$ such that for all $n \geq N$:
$\left|\frac{1}{n} - 0\right| < \varepsilon$

This simplifies to showing $\frac{1}{n} < \varepsilon$.

Given $\varepsilon > 0$, by the Archimedean property, there exists a positive integer $N$ such that $N > \frac{1}{\varepsilon}$, which means $\frac{1}{N} < \varepsilon$.

For any $n \geq N$, we have $n \geq N > \frac{1}{\varepsilon}$, so $\frac{1}{n} \leq \frac{1}{N} < \varepsilon$.

Therefore, $\left|\frac{1}{n} - 0\right| = \frac{1}{n} < \varepsilon$ for all $n \geq N$.

This proves that $\lim_{n \to \infty} \frac{1}{n} = 0$.

\section{Chapter 6: Discrete Mathematics}

\subsection{Exercise 1}
\textbf{Problem:} Prove that in any graph, the number of vertices with odd degree is even.

\textbf{Solution:}
Let $G = (V, E)$ be a graph. Let $V_{\text{odd}}$ be the set of vertices with odd degree and $V_{\text{even}}$ be the set of vertices with even degree.

By the handshaking lemma:
$\sum_{v \in V} \deg(v) = 2|E|$

Since $2|E|$ is even, the sum of all degrees is even.

We can partition this sum:
$\sum_{v \in V} \deg(v) = \sum_{v \in V_{\text{even}}} \deg(v) + \sum_{v \in V_{\text{odd}}} \deg(v)$

The first sum, $\sum_{v \in V_{\text{even}}} \deg(v)$, is a sum of even numbers, so it's even.

Since the total sum is even and the first part is even, the second part $\sum_{v \in V_{\text{odd}}} \deg(v)$ must also be even.

But this is a sum of odd numbers. For a sum of odd numbers to be even, there must be an even number of terms.

Therefore, $|V_{\text{odd}}|$ is even, meaning the number of vertices with odd degree is even.

\section{Chapter 7: Probability Theory}

\subsection{Exercise 2}
\textbf{Problem:} If $X \sim \text{Binomial}(10, 0.3)$, compute $P(X = 3)$ and $E[X]$.

\textbf{Solution:}
For a binomial distribution with parameters $n = 10$ and $p = 0.3$:

$P(X = 3) = \binom{10}{3} (0.3)^3 (0.7)^7$

Computing each part:
- $\binom{10}{3} = \frac{10!}{3! \cdot 7!} = \frac{10 \cdot 9 \cdot 8}{3 \cdot 2 \cdot 1} = 120$
- $(0.3)^3 = 0.027$
- $(0.7)^7 = 0.0823543$

Therefore:
$P(X = 3) = 120 \times 0.027 \times 0.0823543 \approx 0.2668$

For the expected value:
$E[X] = np = 10 \times 0.3 = 3$

\section{Chapter 8: Geometry}

\subsection{Exercise 4}
\textbf{Problem:} A triangle has sides of length 3, 4, and 5. Find its area using both the base-height formula and Heron's formula.

\textbf{Solution:}
First, note that $3^2 + 4^2 = 9 + 16 = 25 = 5^2$, so this is a right triangle with legs of length 3 and 4, and hypotenuse of length 5.

\textbf{Method 1: Base-height formula}
Using the legs as base and height:
$A = \frac{1}{2} \times \text{base} \times \text{height} = \frac{1}{2} \times 3 \times 4 = 6$

\textbf{Method 2: Heron's formula}
First, find the semi-perimeter:
$s = \frac{3 + 4 + 5}{2} = 6$

Then apply Heron's formula:
\begin{align}
A &= \sqrt{s(s-a)(s-b)(s-c)} \\
&= \sqrt{6(6-3)(6-4)(6-5)} \\
&= \sqrt{6 \times 3 \times 2 \times 1} \\
&= \sqrt{36} \\
&= 6
\end{align}

Both methods give the same result: the area is 6 square units.


% Bibliography
\backmatter
\bibliographystyle{plain}
\bibliography{references}

\end{document}