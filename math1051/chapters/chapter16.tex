\chapter{Eigenvalues and Eigenvectors}

Consider a linear transformation represented by a square matrix, denoted $A$. When a vector is transformed by the matrix, its direction and magnitude can change. Consider the unique case of special vectors, called \textbf{eigenvectors}, denoted $\underline{x}$, that only change in magnitude (not direction) under this transformation. The magnitude by which this eigenvector is scaled is called the \textbf{eigenvalue}, denoted $\lambda$.

The relationship between a matrix $A$, its eigenvalues $\lambda$, and eigenvectors $\underline{x}$ is given by the equation:

$$A \underline{x} = \lambda \underline{x}$$

This is known as the \textbf{definition of eigenvalues and eigenvectors}.

\section{Eigenvalue Equation}

To find the eigenvalues $\lambda$ of a matrix $A$, we solve the equation:
$$|A - \lambda I| = 0$$

\section{Eigenvector Equation}

To find the eigenvectors $\underline{x}$ corresponding to a specific eigenvalue $\lambda$, we solve the equation:
$$(A - \lambda I) \underline{x} = \underline{0}$$

\section{Derivation of Eigenvalue and Eigenvector Equations}

To find the eigenvector equation, we can rearrange the definition of eigenvalues and eigenvectors. 

$$
\begin{aligned}
A \underline{x} &= \lambda \underline{x} \\
\iff A \underline{x} &= \lambda I \underline{x} \\
\iff A \underline{x} - \lambda \underline{x} &= \underline{0} \\
\iff (A - \lambda I) \underline{x} &= \underline{0} \\
\end{aligned}
$$

To find the eigenvalue equation, note that eigenvectors are non-zero vectors $\underline{x} \neq \underline{0}$. Therefore, the equation $(A - \lambda I) \underline{x} = \underline{0}$ must have non-trivial solutions. This occurs when the matrix $(A - \lambda I)$ is singular, when its determinant is zero:
$$|A - \lambda I| = 0$$

Therefore, we have derived both the eigenvalue and eigenvector equations from the definition of eigenvalues and eigenvectors.

\newpage

\section{Problems}

\begin{question}
Determine the eigenvalues and corresponding eigenvectors of each matrix given below. For every case, verify that the obtained eigenvalues and eigenvectors satisfy the equation $A \underline{x} = \lambda \underline{x}$.
\end{question}

\begin{enumerate}[label=\alph*)]
    \item $A = \begin{pmatrix} 4 & 1 \\ 2 & 3 \end{pmatrix}$ 
    \item $A = \begin{pmatrix} 0 & -2 \\ 1 & -3 \end{pmatrix}$ 
    \item $A = \begin{pmatrix} -6 & -5 \\ 4 & -2 \end{pmatrix}$ 
    \item $A = \begin{pmatrix} 2 & 0 & 0 \\ 0 & 3 & 4 \\ 0 & 0 & 5 \end{pmatrix}$ 
    \item $A = \begin{pmatrix} 1 & 2 & 0 \\ 0 & 1 & 0 \\ 0 & 0 & 3 \end{pmatrix}$ 
    \item $A = \begin{pmatrix} 0 & 1 & 0 \\ 0 & 0 & 1 \\ 1 & 0 & 0 \end{pmatrix}$ 
\end{enumerate}

\begin{question}
Determine whether the vector $\mathbf{v}$ is a linear combination of the given vectors.
\end{question}
\begin{enumerate}[label=\alph*)]
    \item $\mathbf{v} = \begin{pmatrix}3\\1\end{pmatrix},
    \mathbf{u}_1 = \begin{pmatrix}1\\0\end{pmatrix}, 
    \mathbf{u}_2 = \begin{pmatrix}0\\1\end{pmatrix}$
    \item $\mathbf{v} = \begin{pmatrix}2\\4\end{pmatrix}, 
    \mathbf{u}_1 = \begin{pmatrix}1\\2\end{pmatrix}, 
    \mathbf{u}_2 = \begin{pmatrix}2\\4\end{pmatrix}$
    \item $\mathbf{v} = \begin{pmatrix}1\\2\\3\end{pmatrix}, 
    \mathbf{u}_1 = \begin{pmatrix}1\\0\\0\end{pmatrix}, 
    \mathbf{u}_2 = \begin{pmatrix}0\\1\\0\end{pmatrix}$
    \item $\mathbf{v} = \begin{pmatrix}4\\5\\6\end{pmatrix}, 
    \mathbf{u}_1 = \begin{pmatrix}1\\0\\1\end{pmatrix}, 
    \mathbf{u}_2 = \begin{pmatrix}0\\1\\1\end{pmatrix}, 
    \mathbf{u}_3 = \begin{pmatrix}1\\1\\0\end{pmatrix}$
    \item $\mathbf{v} = \begin{pmatrix}1\\0\\0\end{pmatrix}, 
    \mathbf{u}_1 = \begin{pmatrix}1\\1\\0\end{pmatrix},
    \mathbf{u}_2 = \begin{pmatrix}2\\2\\0\end{pmatrix}, 
    \mathbf{u}_3 = \begin{pmatrix}0\\0\\1\end{pmatrix}$
    \item $\mathbf{v} = \begin{pmatrix}0\\1\\2\\3\end{pmatrix}, 
    \mathbf{u}_1 = \begin{pmatrix}0\\1\\0\\0\end{pmatrix}, 
    \mathbf{u}_2 = \begin{pmatrix}0\\0\\1\\0\end{pmatrix}, 
    \mathbf{u}_3 = \begin{pmatrix}0\\0\\0\\1\end{pmatrix}$
\end{enumerate}

\begin{question}
For each of the following sets of vectors, determine if the set is linearly independent or linearly dependent.
\end{question}
\begin{enumerate}[label=\alph*)]
    \item $\left\{
    \mathbf{v}_1 = \begin{pmatrix}1\\2\end{pmatrix},\
    \mathbf{v}_2 = \begin{pmatrix}3\\4\end{pmatrix}
    \right\}$
    \item $\left\{ 
    \mathbf{v}_1 = \begin{pmatrix}1\\0\end{pmatrix},\ 
    \mathbf{v}_2 = \begin{pmatrix}0\\1\end{pmatrix},\ 
    \mathbf{v}_3 = \begin{pmatrix}1\\1\end{pmatrix}
    \right\}$
    \item $\left\{ 
    \mathbf{v}_1 = \begin{pmatrix}1\\0\\4\end{pmatrix},\ 
    \mathbf{v}_2 = \begin{pmatrix}2\\-1\\5\end{pmatrix},\ 
    \mathbf{v}_3 = \begin{pmatrix}4\\1\\11\end{pmatrix}
    \right\}$
    \item $\left\{ 
    \mathbf{v}_1 = \begin{pmatrix}2\\2\\1\\-3\end{pmatrix},\ 
    \mathbf{v}_2 =\begin{pmatrix}2\\-2\\2\\-5\end{pmatrix},\ 
    \mathbf{v}_3 =\begin{pmatrix}0\\-2\\-2\\1\end{pmatrix},\ 
    \mathbf{v}_4 = \begin{pmatrix}1\\0\\1\\-1\end{pmatrix}
    \right\}$
    \item $\left\{ 
    \mathbf{v}_1 = \begin{pmatrix}1\\2\\3\end{pmatrix},\ 
    \mathbf{v}_2 = \begin{pmatrix}4\\5\\6\end{pmatrix},\ 
    \mathbf{v}_3 = \begin{pmatrix}7\\8\\9\end{pmatrix}
    \right\}$
    \item $\left\{ 
    \mathbf{v}_1 = \begin{pmatrix}1\\1\\1\end{pmatrix},\ 
    \mathbf{v}_2 = \begin{pmatrix}-1\\2\\0\end{pmatrix},\ 
    \mathbf{v}_3 = \begin{pmatrix}3\\-1\\1\end{pmatrix}
    \right\}$
\end{enumerate}

\begin{question}
The vectors $\mathbf{v}, \mathbf{u}, \mathbf{w}$ are linearly independent. Determine whether the following sets of vectors are linearly independent or linearly dependent.
\end{question}
\begin{enumerate}[label=\alph*)]
    \item $\{\mathbf{u} + \mathbf{v}, \mathbf{v} + \mathbf{w}, \mathbf{u} + \mathbf{w}\}$
    \item $\{\mathbf{u} - \mathbf{v}, \mathbf{v} - \mathbf{w}, \mathbf{u} - \mathbf{w}\}$
\end{enumerate}

\begin{question}
Determine whether the following sets of vectors form a vector space.
\end{question}
\begin{enumerate}[label=\alph*)]
    \item $U = \left\{
    \begin{pmatrix}x\\y\\z\end{pmatrix} \in \mathbb{R}^3 \ \middle| \ 3x - y + z = 0
    \right\}$
    \item $V = \left\{
    \begin{pmatrix}x\\y\\z\end{pmatrix} \in \mathbb{R}^3 \ \middle| \ x^2 + y^2 \le z^2
    \right\}$
    \item $W = \left\{
    \begin{pmatrix}x\\y\\z\end{pmatrix} \in \mathbb{R}^3 \ \middle| \ 2y - z = 4
    \right\}$
    \item $X = \left\{
    \begin{pmatrix}x\\y\end{pmatrix} \in \mathbb{R}^2 \ \middle| \ x - y = 0
    \right\}$
    \item $Y = \left\{
    \begin{pmatrix}x\\y\\z\end{pmatrix} \in \mathbb{R}^3 \ \middle| \ x + 2y + 3z = 0
    \right\}$
    \item $Z = \left\{
    \begin{pmatrix}x\\y\end{pmatrix} \in \mathbb{R}^2 \ \middle| \ xy = 1
    \right\}$
\end{enumerate}

\begin{question}
Determine whether the following sets of vectors form a vector space. If they do, find a basis and the dimension of the vector space.
\end{question}
\begin{enumerate}[label=\alph*)]
    \item $A = \left\{
    \begin{pmatrix}x\\y\\z\end{pmatrix} \in \mathbb{R}^3 \ \middle| \ x - y \le z
    \right\}$
    \item $B = \left\{
    \begin{pmatrix}x\\y\\z\end{pmatrix} \in \mathbb{R}^3 \ \middle| \ 3x - y + 2z = 0
    \right\}$
    \item $C = \left\{
    \begin{pmatrix}x\\y\\z\end{pmatrix} \in \mathbb{R}^3 \ \middle| \ 2x + y - z = 0 \ \text{and} \ x - y + z = 0
    \right\}$
    \item $D = \left\{
    \begin{pmatrix}x\\y\\z\end{pmatrix} \in \mathbb{R}^3 \ \middle| \ 2x + y - z = 0 \ \text{or} \ x + 2y + z = 0
    \right\}$
    \item $E = \left\{
    \begin{pmatrix}x\\y\\z\\w\end{pmatrix} \in \mathbb{R}^4 \ \middle| \ x + y + z + w = 0
    \right\}$
    \item $F = \left\{
    \begin{pmatrix}x\\y\\z\end{pmatrix} \in \mathbb{R}^3 \ \middle| \ x^2 + y^2 = z^2
    \right\}$
\end{enumerate}

\newpage

\section{Solutions}
